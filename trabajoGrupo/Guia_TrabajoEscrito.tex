%#############################PREAMBLE#############################################
\documentclass[twoside,11pt,]{article}

\usepackage[spanish]{babel}
\usepackage{graphicx}
\usepackage{float}
\usepackage[skins]{tcolorbox}
\usepackage{titlepic}
\usepackage{fancyhdr}
\usepackage{geometry}

\geometry{a4paper, total={170mm,257mm}, left=20mm, top=25mm,}

\pagestyle{fancy}
\lhead{Cartografia Geotecnica}
\rhead{\thepage}
\cfoot{Guía}
\renewcommand{\headrulewidth}{0.4pt}
\renewcommand{\footrulewidth}{0.4pt}

\graphicspath{{G:/My Drive/FIGURAS/}}

\title {GUÍA PRESENTACIÓN TRABAJO EN GRUPO\\ Mapa Geomorfológico de cuenca}
\author{Prof.: Edier Aristizábal\\[5ex]
\includegraphics[width=10.0cm]{unal2}
}
\date{}

%################################BODY############################################
\begin{document}
\maketitle

\emph {versión}: \today

\section*{INTRODUCCIÓN}
La presente guía corresponde a los objetivos, alcances, actividades y evaluación del trabajo en grupo para la elaboración de un mapa geomorfológico de una cuenca seleccionada como parte del curso de Geomorfología. Tiene como propósito orientar al estudiante en la elaboración clara y precisa de su trabajo.

\subsection{Autores del trabajo} 
Aunque se tenga conformado un grupo de 3 personas, las personas que aparezcan como autores del trabajo deben haber participado activamente. Por lo tanto al final del trabajo se deberá señalar de forma breve como participó cada uno de los integrantes en el trabajo.

\subsection{Elaboración de las memorias}
Luego de la presentación del trabajo se deberá elaborar las memorias del mapa geomorfológico. Donde sea acojan las observaciones realizadas durante la presentación. Estas memorias incluyen tanto la descripción breve de la metodología y los criterios utilizados, como la descripción detallada de cada una de los niveles jerárquicos y unidades homogéneas establecidas. No existe un formato estándar, es de libre elección para el estudiante seleccionar el formato de presentación, por lo tanto se recomienda al estudiante revisar diferentes trabajos elaborados por el SGC, el IDEAM, o trabajos internacionales de mapas geomorfológicos. El estudiante deberá seleccionar o crear su propio formato que le permita transmitir de forma adecuada el trabajo realizado. Esta búsqueda y creación del formato hace parte del trabajo. En muchos casos un excelente trabajo se ve opacado por una mala presentación escrita de los resultados, por lo tanto de forma indirecta la selección e implementación del formato puede impactar considerablemente la nota obtenida, pero esto no significa que exista un formato ideal. El objetivo es poder presentar de forma concisa, coherente y clara los trabajos del resultado desarrollado.


\section{CONTENIDO DEL TRABAJO}
\subsection{Objetivos} 
Estudio y análisis de un caso práctico y real donde el estudiante pueda aplicar y desarrollar los elementos teóricos del curso en cartografía geomorfológica.
\subsection*{Objetivos específicos}
\begin{itemize}
\item Identificar una cuenca con un área entre 5 – 20 km$^2$.
\item Compilar y revisar información secundaria de la cuenca.
\item Establecer la geomorfología y geología regional de la cuenca a partir de información secundaria.
\item Elaborar el mapa geomorfológico de la cuenca de estudio.
\item Elaborar dos cortes geomorfológicos de su cuenca
\item Elaborar un diagrama de bloque de la geomorfología de su cuenca
\item Elaborar las memorias del mapa geomorfológico de la cuenca de estudio.
\end{itemize}

\subsection{Alcances}
La escala de trabajo para la cartografía geomorfológica a desarrollar será 1:10.000. Por lo tanto se recomienda que el área de la cuenca seleccionada deberá estar aproximadamente entre 5 y 20 Km2. Esta área le permitirá el estudiante llegar hasta el nivel de componente geomorfológica. Áreas menores a 5 km2 no se aceptan ya que el detalle o escala debería ser mucho mayor, lo que implica técnicas diferentes. Y áreas mayores a 25km2 exigirán mucho más tiempo. Se recomienda seleccionar una cuenca heterogénea desde el punto de vista fisiográfico, de tal forma que el estudiante tenga un amplio espectro de geoformas y posiblemente ambientes. En caso contrario no le permitirá afianzar y poner en práctica los conocimientos adquiridos durante el curso.
\par Durante el curso se revisarán los tipos de mapas geomorfológicos: analíticos, sintéticos y pragmático. Para grupos conformados por estudiantes de ingeniería geológica se recomienda la elaboración de un mapa geomorfológico analítico, para los grupos conformados por estudiantes de ingeniería ambiental se recomienda un mapa geomorfológico sintético. En cuanto a los mapas geomorfológicos pragmáticos se recomiendan tanto para estudiantes de geología o ambiental.
El grupo deberá obtener cartografía básica de la cuenca, del IGAC a escala 1:10.000 o mapas aster a partir del programa ALOS PALSAR con resolución 12,5 m x 12,5 m. En caso de tener conocimientos de sensores remotos se recomienda el uso de imágenes satelitales y fotografías aéreas, así como de herramientas online y libres como Google Earth. En cualquier caso, la selección de la cuenca debe tener como elemento fundamental la disponibilidad de la cartografía básica.
\par Se recomienda considerar un caso de estudio que el grupo pueda visitar, de tal forma que pueda realizar un levantamiento o verificación de campo. Sin embargo, es solo una recomendación y no una exigencia, ni tampoco un criterio de evaluación.
Las normas de presentación de las memorias son libres. Por lo tanto, se deberá seleccionar una norma y estilo de tal forma que se ajuste a los criterios de evaluación, y que el estudiante considere más adecuado para transmitir el esfuerzo y resultados del trabajo elaborado.
\par El curso de geomorfología es sobre las geoformas y los procesos e historia geológica que dieron lugar a ellas, y no un curso SIG. Las herramientas SIG pueden ser muy útiles, y especialmente en cartografía, pero no son el objeto del curso, y por lo tanto no son un criterio de evaluación.
\subsection{Actividades a realizar}
Para el logro de los objetivos específicos y alcances anteriormente mencionados se deberán realizar las siguientes actividades:\\
\textbf{Selección de la cuenca de estudio}. Se deberá seleccionar una cuenca con un área entre 5 y 20 km2. Para seleccionar la cuenca son muy útiles herramientas de sensores remotos libres en la web, tales como Goole Earth. Existen múltiples páginas donde es posible descargar información de sensores remotos de forma libre, tales como el DEM con resolución espacial de 12,5 m, 30 m, o 90 m del programa ALOS PALSAR. Es importante seleccionar una cuenca que le permita al estudiante evaluar diferentes ambientes y topografías. Si entre las posibilidades del grupo está visitar la zona de campo es necesario revisar las vías de acceso que le permitan no solamente acceder al sitio, sino realizar un recorrido.
\\
\textbf{Compilación y análisis de información secundaria}. Luego de seleccionar la cuenca, el estudiante deberá realizar una exhaustiva búsqueda de información secundaria, inicialmente información general de la cuenca, como localización, centros poblados, actividades económicas, entre otros. Posteriormente se deberá enfocar en información geológica y geomorfológica regional, esto implica revisar fuentes oficiales como el IGAC el IDEAM y el Servicio Geológico Colombiano (SGC). Una alcance importante en esta fase es identificar las escala más detalladas de información geológica y geomorfológica existente de la cuenca, para esto son muy importante fuentes como las Corporaciones Autónomas Regionales (CAR), para Antioquia existe CORANTIOQUIA; CORNARE, CORPOURABA y el AREA METROPOLITANA DEL VALLE DE ABURRÁ. La cuenca seleccionada puede pertenecer a un Plan de Ordenación y Manejo de Cuencas donde seguramente existirá mucha información disponible. Los municipios en sus POT también podrían ser una fuente importante de información, al igual que la gobernación. Y finalmente las universidades a través de tesis o estudios pueden tener información al respecto. Como uno de los alcances fundamentales de esta fase, y que deberá presentarse en el avance, es el contexto geomorfológico de la cuenca a nivel de los órdenes superiores jerárquicos. Para esto se recomienda consultar el estudio de amenaza por movimientos en masa a escala 1:100.000 del SGC, en el cual se elaboraron los mapas geomorfológicos de todas las planchas a escala 1:100.000. En la página del SGC se podrá consultar dicha información. Como esta información se encuentra a escala 1:100.000, es útil para establecer los niveles jerárquicos superiores, pero se deberá ajustar los límites a la escala 1:10.000, ya que el detalle topográfico es mucho mayor, lo implica ajustar los límites de los polígonos al detalle topográfico de las planchas 1:10.000.
\\
\textbf{Elabora el mapa geomorfológico de la cuenca de estudio:} Partiendo del mayor detalle obtenido en los estudios y mapas geomorfológicos de información secundaria, se deberá realizar el mapa geomorfológico a nivel de componente, es decir el máximo detalle geomorfológico de acuerdo con el sistema adoptado por el SGC. Para esta fase es muy importante utilizar herramientas de sensores remotos como los mapas topográficos y DEM, que le permitirán tener una primera aproximación a las formas del terreno, al igual que imágenes de satélite o fotografías aéreas que pueden brindar información adicional de acuerdo con el rango del espectro electromagnético que cubra. Una herramienta de fácil acceso y que brinda esta información es Google Earth. La teórica sobre estas herramientas será dictada durante el curso.
A partir del DEM el estudiante podrá explorar herramientas con programas SIG como ArcGIS, QGIS, entre otros para obtener información morfométrica que le ayuda a elaborar el mapa geomorfológico. Sin embargo es importante tener en claro que el objetivo final no son los mapas morfométricos de la cuenca como pendientes, aspectos, curvatura, entre otros, los cuales se obtienen directamente del DEM. Estos mapas son solo herramientas para la elaboración del mapa geomorfológico.
En este sentido es importante utilizar otras herramientas como imágenes de satélite o fotografías aéreas. Para estas última es necesario conocer la técnica de fotointerpretación que se dicta en el curso de Sensores Remotos. Con dichas fotografías aéreas y equipos para fotointerpretación denominados estereoscopios, disponibles en la universidad, se tiene un acercamiento en tres dimensiones a toda la cuenca, lo cual es fundamental para la cartografía. Independiente si el estudiante ha visto o no dicho curso pueden participar de la monitorias del curso y a través del monitor obtener herramientas iniciales que le ayuden en este nivel a elaborar el mapa geomorfológico de la cuenca. Para esta labor también se recomienda utilizar el DEM con la ortofoto superpuesta en un ambiente SIG. Esto le permitirá tener una visión digital de la cuenca en tres dimensiones y diferentes vistas para elaborar el mapa geomorfológico. Para esto deberán consultar en la web o con personas expertas en SIG. 
Como pare de este ítem se deberá elaborar tabla resumen de los diferentes niveles geomorfológicos, los criterios utilizados y características de las geoformas en la cuenca.
\\
\textbf{Elaborar cortes geomorfológicos}. Se deberán elaborar dos (2) cortes geomorfológicos de la cuenca. Uno longitudinal al eje principal de la cuenca, y otro transversal. Dichos cortes deberán contener información de los diferentes niveles geomorfológicos, así como de los procesos morfodinámicos y demás información relevante que considere el grupo. También se deberá presentar información geológica y de suelos del corte geomorfológico.
\\
\textbf{Elaborar diagrama de bloque}. Se deberá elaborar un diagrama de bloque representativo de la cuenca, donde se observen los diferentes procesos morfodinámicos que actuan en la cuenca y geoforma características.

\end{document}