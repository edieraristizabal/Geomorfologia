\documentclass{beamer}

\usepackage{beamerthemeCambridgeUS}
\usepackage{textpos}
\usepackage{ragged2e}

\graphicspath{{G:/My Drive/FIGURAS/}}

\title[Ambiente Volcánico]{GEOMORFOLOGÍA}
\author[Edier Aristizabal]{Edier V. Aristizábal G.}
\institute{evaristizabalg@unal.edu.co}
\date{Versión:\today}

\addtobeamertemplate{headline}{}{%
	\begin{textblock*}{2mm}(.9\textwidth,0cm)
	\hfill\includegraphics[height=1cm]{un}  
	\end{textblock*}}

\begin{document}
%%%%%%%%%%%%%%%%%%%%%%%%%%%%%%%%%%%%%%%%%%%%%%%%%%%%%%%%%%%%%%%%%%%%%%%%%
\begin{frame}
\titlepage
\centering
	\includegraphics[width=5cm]{unal}\hspace*{4.75cm}~%
   	\includegraphics[width=2cm]{logo3} 
\end{frame}
%%%%%%%%%%%%%%%%%%%%%%%%%%%%%%%%%%%%%%%%%%%%%%%%%%%%%%%%%%%%%%%%%%%%%%%%%
\begin{frame}
\frametitle{Ambiente Volcánico}
\begin{center}
\includegraphics[scale=0.55]{volcanico}
\end{center}
\end{frame}
%%%%%%%%%%%%%%%%%%%%%%%%%%%%%%%%%%%%%%%%%%%%%%%%%%%%%%%%%%%%%%%%%%%%%%%
\begin{frame}
\frametitle{Ambiente Tectónico de Volcánico}
\begin{center}
\includegraphics[scale=0.50]{volcanico2}
\end{center}
\tiny{Fuente: Manville et al (2009)}
\end{frame}
%%%%%%%%%%%%%%%%%%%%%%%%%%%%%%%%%%%%%%%%%%%%%%%%%%%%%%%%%%%%%%%%%%%%%%%
\begin{frame}
\frametitle{Dinámica de erupciones}
\begin{center}
\includegraphics[scale=0.6]{dinamicaerupciones}
\end{center}
\end{frame}
%%%%%%%%%%%%%%%%%%%%%%%%%%%%%%%%%%%%%%%%%%%%%%%%%%%%%%%%%%%%%%%%%%%%%%%
\begin{frame}
\frametitle{Dinámica de erupciones}
\begin{center}
\includegraphics[scale=0.45]{tablevolcanos}
\end{center}
\tiny{\url{http://intheplaygroundofgiants.com/geology-of-central-oregon/the-geology-of-volcanoes-and-volcanism/}}
\end{frame}
%%%%%%%%%%%%%%%%%%%%%%%%%%%%%%%%%%%%%%%%%%%%%%%%%%%%%%%%%%%%%%%%%%%%%%%
\begin{frame}
\begin{center}
\includegraphics[scale=0.58]{erupciones}
\end{center}
\end{frame}
%%%%%%%%%%%%%%%%%%%%%%%%%%%%%%%%%%%%%%%%%%%%%%%%%%%%%%%%%%%%%%%%%%%%%%%
\begin{frame}
\frametitle{Geoformas Volcánicas}
\begin{center}
\includegraphics[scale=0.45]{volcanicgeoformas}
\end{center}
\end{frame}
%%%%%%%%%%%%%%%%%%%%%%%%%%%%%%%%%%%%%%%%%%%%%%%%%%%%%%%%%%%%%%%%%%%%%%%
\begin{frame}
\frametitle{Basado en el edificio volcánico}
\begin{center}
\includegraphics[scale=0.6]{edificiovolcanico}
\end{center}
\end{frame}
%%%%%%%%%%%%%%%%%%%%%%%%%%%%%%%%%%%%%%%%%%%%%%%%%%%%%%%%%%%%%%%%%%%%%%%
\begin{frame}
\frametitle{Basado en el edificio volcánico}
\begin{center}
\includegraphics[scale=0.55]{edificiovolcanico2}
\end{center}
\end{frame}
%%%%%%%%%%%%%%%%%%%%%%%%%%%%%%%%%%%%%%%%%%%%%%%%%%%%%%%%%%%%%%%%%%%%%%%
\begin{frame}
\frametitle{Basado en el modo de erupción}
\begin{center}
\includegraphics[scale=0.50]{tipoerupcion}
\end{center}
\end{frame}
%%%%%%%%%%%%%%%%%%%%%%%%%%%%%%%%%%%%%%%%%%%%%%%%%%%%%%%%%%%%%%%%%%%%%%%
\begin{frame}
\frametitle{Geoformas Monogenéticas vs Poligenéticas}
\begin{center}
\includegraphics[scale=0.70]{monogenetica}
\end{center}
\end{frame}
%%%%%%%%%%%%%%%%%%%%%%%%%%%%%%%%%%%%%%%%%%%%%%%%%%%%%%%%%%%%%%%%%%%%%%%
\begin{frame}
\begin{figure}
\begin{center}
\includegraphics[scale=0.65]{sierragrande}
\caption{Sierra Grande (New México, USA)}
\end{center}
\end{figure}
\end{frame}
%%%%%%%%%%%%%%%%%%%%%%%%%%%%%%%%%%%%%%%%%%%%%%%%%%%%%%%%%%%%%%%%%%%%%%%
\begin{frame}
\begin{figure}
\begin{center}
\includegraphics[scale=0.48]{domosantaelena}
\caption{Domo de lava creciendo dentro del cráter de Mt. St. Helens después de la erupción de 1980)}
\end{center}
\end{figure}
\end{frame}
%%%%%%%%%%%%%%%%%%%%%%%%%%%%%%%%%%%%%%%%%%%%%%%%%%%%%%%%%%%%%%%%%%%%%%%
\begin{frame}
\frametitle{Cono de Escoria}
\begin{figure}
\begin{center}
\includegraphics[scale=0.42]{conoescoria}
\end{center}
\end{figure}
\tiny{\url{https://en.wikipedia.org/wiki/Cinder_cone}}
\end{frame}
%%%%%%%%%%%%%%%%%%%%%%%%%%%%%%%%%%%%%%%%%%%%%%%%%%%%%%%%%%%%%%%%%%%%%%%
\begin{frame}
\frametitle{Cono de Escoria(\emph{cinder cone})}
\begin{figure}
\begin{center}
\includegraphics[scale=0.7]{cindercone}
\end{center}
\end{figure}
\end{frame}
%%%%%%%%%%%%%%%%%%%%%%%%%%%%%%%%%%%%%%%%%%%%%%%%%%%%%%%%%%%%%%%%%%%%%%%
\begin{frame}
\frametitle{\emph{Tuff ring}}
\begin{figure}
\begin{center}
\includegraphics[scale=0.65]{tuffring}
\caption{Diamond Head Crater, Honolulu (Hawaii)}
\end{center}
\end{figure}
\end{frame}
%%%%%%%%%%%%%%%%%%%%%%%%%%%%%%%%%%%%%%%%%%%%%%%%%%%%%%%%%%%%%%%%%%%%%%%
\begin{frame}
\frametitle{Estratovolcano}
\begin{figure}
\begin{center}
\includegraphics[scale=0.65]{estratovolcano}
\caption{Estratovolcán del Arenal en erupción (Costa Rica)}
\end{center}
\end{figure}
\tiny{tomada de \url{http://www.biodiversidadvirtual.org/geologia/Estratovolcan-del-Arenal-en-erupcion-img2832.html}}
\end{frame}
%%%%%%%%%%%%%%%%%%%%%%%%%%%%%%%%%%%%%%%%%%%%%%%%%%%%%%%%%%%%%%%%%%%%%%%
\begin{frame}
\frametitle{Estratovolcano}
\begin{figure}
\begin{center}
\includegraphics[scale=0.55]{estratovolcano1}
\caption{Estratovolcán del Monte Fuji (Japón)}
\end{center}
\end{figure}
\end{frame}
%%%%%%%%%%%%%%%%%%%%%%%%%%%%%%%%%%%%%%%%%%%%%%%%%%%%%%%%%%%%%%%%%%%%%%%
\begin{frame}
\begin{figure}
\begin{center}
\includegraphics[scale=0.4]{estratovsescudo}
\end{center}
\end{figure}
\end{frame}
%%%%%%%%%%%%%%%%%%%%%%%%%%%%%%%%%%%%%%%%%%%%%%%%%%%%%%%%%%%%%%%%%%%%%%%
\begin{frame}
\frametitle{Cuellos volcánicos}
\begin{columns}
	\begin{column}{.5\linewidth}
	\begin{center}
	\includegraphics[scale=0.55]{cuello}
	\end{center}
	\end{column}
	\begin{column}{.5\linewidth}
	\begin{center}
	\includegraphics[scale=0.45]{cuello1}\vfill
	\end{center}
	\end{column}
\end{columns}
\end{frame}
%%%%%%%%%%%%%%%%%%%%%%%%%%%%%%%%%%%%%%%%%%%%%%%%%%%%%%%%%%%%%%%%%%%%%%%
\begin{frame}
\frametitle{Crateres Maar}
\begin{figure}
\begin{center}
\includegraphics[scale=0.65]{maar}
\end{center}
\end{figure}
\tiny{Source:USGS}
\end{frame}
%%%%%%%%%%%%%%%%%%%%%%%%%%%%%%%%%%%%%%%%%%%%%%%%%%%%%%%%%%%%%%%%%%%%%%%
\begin{frame}
\frametitle{Crateres Maar}
\begin{figure}
\begin{center}
\includegraphics[scale=0.25]{maar1}
\end{center}
\caption{Views of East Ukinrek Maar Crater, which formed in April, 1977 during a 10-day eruption}
\end{figure}
\tiny{https://geology.com/stories/13/maar/}
\end{frame}
%%%%%%%%%%%%%%%%%%%%%%%%%%%%%%%%%%%%%%%%%%%%%%%%%%%%%%%%%%%%%%%%%%%%%%%
\begin{frame}
\frametitle{Crateres Maar}
\begin{figure}
\begin{center}
\includegraphics[scale=0.7]{maar2}
\end{center}
\end{figure}
\tiny{https://geology.com/stories/13/maar/}
\end{frame}
%%%%%%%%%%%%%%%%%%%%%%%%%%%%%%%%%%%%%%%%%%%%%%%%%%%%%%%%%%%%%%%%%%%%%%%
\begin{frame}
\frametitle{Campos de Lava}
\begin{figure}
\begin{center}
\includegraphics[scale=0.75]{camposlava}
\end{center}
\end{figure}
\end{frame}
%%%%%%%%%%%%%%%%%%%%%%%%%%%%%%%%%%%%%%%%%%%%%%%%%%%%%%%%%%%%%%%%%%%%%%%
\begin{frame}
\frametitle{Calderas}
\begin{figure}
\begin{center}
\includegraphics[scale=0.6]{calderas}
\end{center}
\end{figure}
\end{frame}
%%%%%%%%%%%%%%%%%%%%%%%%%%%%%%%%%%%%%%%%%%%%%%%%%%%%%%%%%%%%%%%%%%%%%%%
\begin{frame}
\frametitle{Calderas}
\begin{figure}
\begin{center}
\includegraphics[scale=0.6]{calderas1}
\end{center}
\caption{Calderas Taal (Filipinas)}
\end{figure}
\end{frame}
%%%%%%%%%%%%%%%%%%%%%%%%%%%%%%%%%%%%%%%%%%%%%%%%%%%%%%%%%%%%%%%%%%%%%%%
\begin{frame}
\frametitle{Calderas}
\begin{figure}
\begin{center}
\includegraphics[scale=0.6]{calderasomma}
\end{center}
\end{figure}
\end{frame}
%%%%%%%%%%%%%%%%%%%%%%%%%%%%%%%%%%%%%%%%%%%%%%%%%%%%%%%%%%%%%%%%%%%%%%%
\begin{frame}
\frametitle{Calderas Colapso - Resurgimiento}
\begin{figure}
\begin{center}
\includegraphics[scale=0.6]{caldera2}
\end{center}
\end{figure}
\end{frame}
%%%%%%%%%%%%%%%%%%%%%%%%%%%%%%%%%%%%%%%%%%%%%%%%%%%%%%%%%%%%%%%%%%%%%%%
\begin{frame}
\frametitle{Caldera Krakatoa}
\begin{figure}
\begin{center}
\includegraphics[scale=0.6]{calderakrakatoa}
\end{center}
\end{figure}
\end{frame}
%%%%%%%%%%%%%%%%%%%%%%%%%%%%%%%%%%%%%%%%%%%%%%%%%%%%%%%%%%%%%%%%%%%%%%%
\begin{frame}
\frametitle{Caldera Krakatoa}
\begin{figure}
\begin{center}
\includegraphics[scale=0.5]{krakatoa}
\end{center}
\end{figure}
\end{frame}
%%%%%%%%%%%%%%%%%%%%%%%%%%%%%%%%%%%%%%%%%%%%%%%%%%%%%%%%%%%%%%%%%%%%%%%
\begin{frame}
\frametitle{Caldera Krakatoa}
\begin{figure}
\begin{center}
\includegraphics[scale=0.5]{krakatoa1}
\end{center}
\end{figure}
\end{frame}
%%%%%%%%%%%%%%%%%%%%%%%%%%%%%%%%%%%%%%%%%%%%%%%%%%%%%%%%%%%%%%%%%%%%%%%
\begin{frame}
\frametitle{Caldera Krakatoa}
\begin{figure}
\begin{center}
\includegraphics[scale=0.4]{krakatoa2}
\end{center}
\end{figure}
\end{frame}
%%%%%%%%%%%%%%%%%%%%%%%%%%%%%%%%%%%%%%%%%%%%%%%%%%%%%%%%%%%%%%%%%%%%%%%
\begin{frame}
\frametitle{Caldera Krakatoa}
\begin{figure}
\begin{center}
\includegraphics[scale=0.4]{krakatoa3}
\end{center}
\end{figure}
\end{frame}
%%%%%%%%%%%%%%%%%%%%%%%%%%%%%%%%%%%%%%%%%%%%%%%%%%%%%%%%%%%%%%%%%%%%%%%
\begin{frame}
\frametitle{Depósitos de origen Volcánico}
\begin{figure}
\begin{center}
\includegraphics[scale=0.4]{depositosvolcanicos}
\end{center}
\end{figure}
\end{frame}
%%%%%%%%%%%%%%%%%%%%%%%%%%%%%%%%%%%%%%%%%%%%%%%%%%%%%%%%%%%%%%%%%%%%%%%
\begin{frame}
\frametitle{Depósitos de origen Volcánico}
\justifying
Los sedimentos y depósitos de origen volcánico pueden ser divididos de acuerdo a su origen en:\vfill
\textbf{Piroclásticos} $\rightarrow$ primarios $\rightarrow$ restringido a material generado, transportado y depositado por vulcanismo explosivo subaereo (Explosivo).\vfill
\textbf{Hyaloclásticos} $\rightarrow$ primarios $\rightarrow$ Fragmentos formados por el choque térmico cuando la lava caliente entra en contacto con el agua fría (efusivo).\vfill
\textbf{Autoclásticos} $\rightarrow$ primarios $\rightarrow$ formados por movimiento mecánico o gravitacional de flujos de lava y/o domos (efusivo).\vfill
\textbf{Epiclásticos} $\rightarrow$ secundarios $\rightarrow$ fragmentos volcánicos que son producidos por erosión de rocas volcánicas por viento, agua o hielo de rocas volcánicas preexistentes consolidadas.
\end{frame}
%%%%%%%%%%%%%%%%%%%%%%%%%%%%%%%%%%%%%%%%%%%%%%%%%%%%%%%%%%%%%%%%%%%%%%%
\begin{frame}
\frametitle{Depósitos Piroclásticos}
\framesubtitle{Teffra}
\justifying
\textbf{Teffra}: Material expulsado, fragmentado y distribuido por el viento, no compactado se denomina tefra, independientemente de la composición o del tamaño de los granos. Los diferentes fragmentos, sueltos o compactados, son llamados piroclástos.
\begin{figure}
\begin{center}
\includegraphics[scale=0.6]{teffra}
\end{center}
\end{figure}
\end{frame}
%%%%%%%%%%%%%%%%%%%%%%%%%%%%%%%%%%%%%%%%%%%%%%%%%%%%%%%%%%%%%%%%%%%%%%%
\begin{frame}
\frametitle{Depósitos Piroclásticos}
\begin{figure}
\begin{center}
\includegraphics[scale=0.6]{teffra1}
\end{center}
\end{figure}
\end{frame}
%%%%%%%%%%%%%%%%%%%%%%%%%%%%%%%%%%%%%%%%%%%%%%%%%%%%%%%%%%%%%%%%%%%%%%%
\begin{frame}
\frametitle{Depósitos Piroclásticos}
\begin{figure}
\begin{center}
\includegraphics[scale=0.5]{piro}
\end{center}
\end{figure}
\end{frame}
%%%%%%%%%%%%%%%%%%%%%%%%%%%%%%%%%%%%%%%%%%%%%%%%%%%%%%%%%%%%%%%%%%%%%%%
\begin{frame}
\frametitle{Flujos Piroclásticos}
\begin{figure}
\begin{center}
\includegraphics[scale=0.5]{flow}
\end{center}
\end{figure}
\end{frame}
%%%%%%%%%%%%%%%%%%%%%%%%%%%%%%%%%%%%%%%%%%%%%%%%%%%%%%%%%%%%%%%%%%%%%%%
\begin{frame}
\frametitle{Flujos}
\begin{figure}
\begin{center}
\includegraphics[scale=0.6]{flujos}
\end{center}
\end{figure}
\end{frame}
%%%%%%%%%%%%%%%%%%%%%%%%%%%%%%%%%%%%%%%%%%%%%%%%%%%%%%%%%%%%%%%%%%%%%%%
\begin{frame}
\frametitle{Nube Ardiente}
\begin{figure}
\begin{center}
\includegraphics[scale=0.6]{nube}
\end{center}
\end{figure}
\end{frame}
%%%%%%%%%%%%%%%%%%%%%%%%%%%%%%%%%%%%%%%%%%%%%%%%%%%%%%%%%%%%%%%%%%%%%%%
\begin{frame}
\frametitle{Lahar}
\begin{figure}
\begin{center}
\includegraphics[scale=0.4]{lahar}
\end{center}
\end{figure}
\end{frame}
%%%%%%%%%%%%%%%%%%%%%%%%%%%%%%%%%%%%%%%%%%%%%%%%%%%%%%%%%%%%%%%%%%%%%%%
\begin{frame}
\frametitle{Geoformas Plutónicas}
\begin{figure}
\begin{center}
\includegraphics[scale=0.5]{geoformasplutonicas}
\end{center}
\end{figure}
\end{frame}
%%%%%%%%%%%%%%%%%%%%%%%%%%%%%%%%%%%%%%%%%%%%%%%%%%%%%%%%%%%%%%%%%%%%%%%
\begin{frame}
\frametitle{Geoformas Plutónicas}
\framesubtitle{Lacolitos}
\small{Cuerpos intrusivos que generan formas dómicas sobre los techos de la roca encajante, son usualmente ácidos y se generan comparativamente superficiales en áreas relativamente  poca afectadas (Twidale, 1971)}.\\
\begin{figure}
\begin{center}
\includegraphics[scale=0.42]{lacolitos}
\end{center}
\end{figure}
\end{frame}
%%%%%%%%%%%%%%%%%%%%%%%%%%%%%%%%%%%%%%%%%%%%%%%%%%%%%%%%%%%%%%%%%%%%%%%
\begin{frame}
\frametitle{Geoformas Plutónicas}
\framesubtitle{Lacolitos, lopolitos y facolitos}
\begin{figure}
\begin{center}
\includegraphics[scale=0.38]{lopolitos}
\end{center}
\end{figure}
\end{frame}
%%%%%%%%%%%%%%%%%%%%%%%%%%%%%%%%%%%%%%%%%%%%%%%%%%%%%%%%%%%%%%%%%%%%%%%
\begin{frame}
\frametitle{Dikes}
\small{Cuerpos intrusivos tabulares, verticales o cercanos, usualmente cortando la roca encajante (Twidale, 1971).}
\begin{figure}
\begin{center}
\includegraphics[scale=0.45]{dike}
\end{center}
\end{figure}
\end{frame}
%%%%%%%%%%%%%%%%%%%%%%%%%%%%%%%%%%%%%%%%%%%%%%%%%%%%%%%%%%%%%%%%%%%%%%%
\begin{frame}
\frametitle{Sill}
\small{Masas tabulares emplazadas horizontalmente y usualmente paralelas a la estratificación, clivaje o foliacion de la roca encajante (Twidale, 1971).}
\begin{figure}
\begin{center}
\includegraphics[scale=0.45]{sill}
\end{center}
\end{figure}
\end{frame}
%%%%%%%%%%%%%%%%%%%%%%%%%%%%%%%%%%%%%%%%%%%%%%%%%%%%%%%%%%%%%%%%%%%%%%%
\end{document}