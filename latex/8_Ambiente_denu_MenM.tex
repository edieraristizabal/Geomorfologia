\documentclass{beamer}

\usepackage{beamerthemeCambridgeUS}
\usepackage{textpos}
\usepackage{ragged2e}
\usepackage{ulem}

\graphicspath{{G:/My Drive/FIGURAS/}}

\title[Ambiente Denudacional: MenM]{GEOMORFOLOGÍA}
\author[Edier Aristizabal]{Edier V. Aristizábal G.}
\institute{evaristizabalg@unal.edu.co}
\date{\tiny{Versión:\today}}

\addtobeamertemplate{headline}{}{%
	\begin{textblock*}{2mm}(.9\textwidth,0cm)
	\hfill\includegraphics[height=1cm]{un}  
	\end{textblock*}}

\begin{document}
%%%%%%%%%%%%%%%%%%%%%%%%%%%%%%%%%%%%%%%%%%%%%%%%%%%%%%%%%%%%%%%%%%%%%%%%%
\begin{frame}
\titlepage
\centering
	\includegraphics[width=5cm]{unal}\hspace*{4.75cm}~%
   	\includegraphics[width=2cm]{logo3} 
\end{frame}
%%%%%%%%%%%%%%%%%%%%%%%%%%%%%%%%%%%%%%%%%%%%%%%%%%%%%%%%%%%%%%%%%%%%%%%%%
\begin{frame}
\centering
   	\includegraphics[scale=0.55]{slope} 
\end{frame}
%%%%%%%%%%%%%%%%%%%%%%%%%%%%%%%%%%%%%%%%%%%%%%%%%%%%%%%%%%%%%%%%%%%%%%%%%
\begin{frame}
\frametitle{Ladera (\it{hillslope})}
\small{Elemento básico de todo paisaje, gran variedad de tamaños y formas, resultado de los procesos de laderas.\\
\textbf{Definición}:Unidad de geoforma inclinada con un ángulo de pendiente mayor que un umbral mínimo que lo delimita de llanuras y menor a un umbral máximo que lo delimita de superficies verticales, y que es limitado por una unidad de geoforma en la parte superior e inferior (Denh et al., 2001).\\\textbf{Propiedades fundamentales para su definición}: (i) geometría local, (ii) relación con las geoformas externas, (iii) escala, (iv) proceso asociados.}
\centering
   	\includegraphics[scale=0.50]{ladera} 
\end{frame}
%%%%%%%%%%%%%%%%%%%%%%%%%%%%%%%%%%%%%%%%%%%%%%%%%%%%%%%%%%%%%%%%%%%%%%%%%
\begin{frame}
\frametitle{Rango de MenM}
\centering
   	\includegraphics[scale=0.55]{rangoMenM} 
\tiny{Tomado de Guzzetti (2005)}
\end{frame}
%%%%%%%%%%%%%%%%%%%%%%%%%%%%%%%%%%%%%%%%%%%%%%%%%%%%%%%%%%%%%%%%%%%%%%%%%
\begin{frame}
\frametitle{Formas y Partes}
\centering
   	\includegraphics[scale=0.50]{formasmenm} 
\tiny{Fuente: GEMMA (2007) y Cruden  \& Varnes (1996)}
\end{frame}
%%%%%%%%%%%%%%%%%%%%%%%%%%%%%%%%%%%%%%%%%%%%%%%%%%%%%%%%%%%%%%%%%%%%%%%%%
\begin{frame}
\frametitle{Términos utilizados para MenM}
\begin{center}
   	\includegraphics[scale=0.55]{terminosMenM} 
\end{center}
\tiny{Tomado de GEMMA (2007) y Cruden  \& Varnes (1996)}
\end{frame}
%%%%%%%%%%%%%%%%%%%%%%%%%%%%%%%%%%%%%%%%%%%%%%%%%%%%%%%%%%%%%%%%%%%%%%%%%
\begin{frame}
\frametitle{Reología de MenM}
\centering
   	\includegraphics[scale=0.55]{reologiaMenM} 
\tiny{Tomado de Boggs(2001)}
\end{frame}
%%%%%%%%%%%%%%%%%%%%%%%%%%%%%%%%%%%%%%%%%%%%%%%%%%%%%%%%%%%%%%%%%%%%%%%%%
\begin{frame}
\frametitle{Definiciones}
Consideraciones mas complejas…
\vfill
\justifying
Sheidegger utiliza la teoría del caos para explicar la ocurrencia, frecuencia y tamaño de los movimientos en masa. \uline{Sistemas complejos abiertos} no lineales desarrollan estados cuasi-estacionarios en el limite del caos. Los movimientos en masa son el resultado de cambios repentinos en el comportamiento a largo plazo generados por pequeñas modificaciones en las condiciones iniciales. Asumiendo la adición de masa endogénica constante, el número de movimientos en masa en cualquier intervalo de tiempo es una función del tamaño, la cual debe seguir una función de distribución: \uline{la frecuencia promedio de una cierta magnitud es inversamente proporcional a cierta relación de su tamaño}.
\vfill
\textbf{Definición}: Modificaciones del terreno dentro del ciclo geomorfológico continuo, y que corresponden a la respuesta normal del sistema debido a complejos parámetros exogénicos (meteóricos) y endogénicos (tectónicos). Sheidegger (1998) 
\end{frame}
%%%%%%%%%%%%%%%%%%%%%%%%%%%%%%%%%%%%%%%%%%%%%%%%%%%%%%%%%%%%%%%%%%%%%%%%%%%
\begin{frame}
\frametitle{Definiciones}
Consideraciones mas complejas…
\vfill
\justifying
Los procesos de movimiento en masa son el resultado de las \uline{condiciones naturales del terreno}, tales como geomorfología, hidrología y geología, y las \uline{modificaciones de estas condiciones} por procesos geodinámicos, vegetación, usos del suelo y actividades humanas. Dichas modificaciones \uline{activan movimientos lentos}, generalmente imperceptibles debido a que las propiedades mecánicas del material o \uline{condiciones de equilibrio decrecen gradualmente}, y posteriormente, factores como precipitación, sismicidad o cortes de origen antrópico \uline{detonan dichos movimientos lentos en rápidos movimientos en masa} (Soeters \& van Westen, 1996).
\vfill
\includegraphics[scale=0.7]{menm}
\end{frame}
%%%%%%%%%%%%%%%%%%%%%%%%%%%%%%%%%%%%%%%%%%%%%%%%%%%%%%%%%%%%%%%%%%%%%%%%%%%
\begin{frame}
\frametitle{Variables Condicionantes \& Detonantes}
\small{
\justifying
\textbf{Variables condicionantes (preparatorias, cuasi-estaticas)}: las cuales hacen la ladera susceptible a fallar sin siquiera iniciarse  y sin embargo tienden a ubicar la ladera en un estado estable marginal: geología, pendiente, aspecto, elevación, propiedades geotécnicas del suelo, vegetación, y patrones de drenaje de largo plazo y meteorización.
\vfill
\textbf{Variables detonantes (dinámicas)}: las cuales cambian la ladera de una estabilidad marginal a un estado inestable y por lo tanto iniciando una falla en un área de determinada susceptibilidad, tales como lluvias intensas, sismos, deshielo, intervención antrópica.\\
\tiny{Fuente: Dai et al, (2002)}
}
\begin{center}
   	\includegraphics[scale=0.40]{variablesmenm} 
\end{center}
\tiny{Tomado de Crozier (1989)}
\end{frame}
%%%%%%%%%%%%%%%%%%%%%%%%%%%%%%%%%%%%%%%%%%%%%%%%%%%%%%%%%%%%%%%%%%%%%%%%%
\begin{frame}
\frametitle{Variables Condicionantes \& Detonantes}
\begin{center}
   	\includegraphics[scale=0.50]{variablesmenm1} 
\end{center}
\tiny{Modificado de Brunsden (2002)}
\end{frame}
%%%%%%%%%%%%%%%%%%%%%%%%%%%%%%%%%%%%%%%%%%%%%%%%%%%%%%%%%%%%%%%%%%%%%%%%%
\begin{frame}
\frametitle{Fuerzas que actuan sobre la ladera}
Las laderas generalmente fallan por cizalla:
\begin{center}
   	\includegraphics[scale=0.50]{cizallamenm} 
\end{center}
\end{frame}
%%%%%%%%%%%%%%%%%%%%%%%%%%%%%%%%%%%%%%%%%%%%%%%%%%%%%%%%%%%%%%%%%%%%%%%%%
\begin{frame}
\frametitle{Fuerzas que actuan sobre la ladera}
Las laderas generalmente fallan por cizalla:
\begin{center}
   	\includegraphics[scale=0.50]{cizallamenm1} 
\end{center}
\end{frame}
%%%%%%%%%%%%%%%%%%%%%%%%%%%%%%%%%%%%%%%%%%%%%%%%%%%%%%%%%%%%%%%%%%%%%%%%%
\begin{frame}
\frametitle{Fuerzas que actuan sobre la ladera}
\begin{center}
   	\includegraphics[scale=0.50]{cizallamenm2} 
\end{center}
\end{frame}
%%%%%%%%%%%%%%%%%%%%%%%%%%%%%%%%%%%%%%%%%%%%%%%%%%%%%%%%%%%%%%%%%%%%%%%%%
\begin{frame}
\frametitle{Ecuación de Coulomb}
Coulomb (1776) $\rightarrow$ resistencia al corte en \uline{términos de esfuerzos totales}.
\begin{center}
   	\includegraphics[scale=0.50]{coulomb} 
\end{center}
\end{frame}
%%%%%%%%%%%%%%%%%%%%%%%%%%%%%%%%%%%%%%%%%%%%%%%%%%%%%%%%%%%%%%%%%%%%%%%%%
\begin{frame}
\frametitle{Principio de esfuerzos efectivos}
Terzagui (1936) $\rightarrow$ resistencia al corte en \uline{términos de esfuerzos efectivos}.
\begin{center}
\it{La resistencia al corte de un suelo depende de la magnitud de los esfuerzos efectivos que soporta el esqueleto del suelo.}
\vfill
   	\includegraphics[scale=0.50]{terzagui} 
\end{center}
\end{frame}
%%%%%%%%%%%%%%%%%%%%%%%%%%%%%%%%%%%%%%%%%%%%%%%%%%%%%%%%%%%%%%%%%%%%%%%%%
\begin{frame}
\frametitle{Principio de esfuerzos efectivos}
El esfuerzo total transmitido a un suelo se divide en dos partes:
\begin{itemize}
\item Dentro del esqueleto del suelo que resultan de las fuerzas que actúan sobre los puntos de contacto entre partículas individuales – \textbf{Esfuerzos efectivos}
\item Dentro del fluido intersticial que ocupa los poros – \textbf{Presiones de poros}
\end{itemize}
\vfill
\begin{center}
   	\includegraphics[scale=0.50]{precionesporos} 
\end{center}
\end{frame}
%%%%%%%%%%%%%%%%%%%%%%%%%%%%%%%%%%%%%%%%%%%%%%%%%%%%%%%%%%%%%%%%%%%%%%%%%
\begin{frame}
\frametitle{Efectos del agua}
\begin{center}
   	\includegraphics[scale=0.45]{aguaenelsuelo} 
\end{center}
\end{frame}
%%%%%%%%%%%%%%%%%%%%%%%%%%%%%%%%%%%%%%%%%%%%%%%%%%%%%%%%%%%%%%%%%%%%%%%%%
\begin{frame}
\frametitle{Zona Vadosa}
\begin{center}
   	\includegraphics[scale=0.52]{vadosa} 
\end{center}
\end{frame}
%%%%%%%%%%%%%%%%%%%%%%%%%%%%%%%%%%%%%%%%%%%%%%%%%%%%%%%%%%%%%%%%%%%%%%%%%
\begin{frame}
\frametitle{Análisis de talud infinito}
\begin{center}
   	\includegraphics[scale=0.50]{taludinfinito} 
\end{center}
\end{frame}
%%%%%%%%%%%%%%%%%%%%%%%%%%%%%%%%%%%%%%%%%%%%%%%%%%%%%%%%%%%%%%%%%%%%%%%%%
\begin{frame}
\frametitle{Clasificaciones}
\begin{center}
   	\includegraphics[scale=0.6]{clasificacionesmenm} 
\end{center}
\end{frame}
%%%%%%%%%%%%%%%%%%%%%%%%%%%%%%%%%%%%%%%%%%%%%%%%%%%%%%%%%%%%%%%%%%%%%%%%%
\begin{frame}
\frametitle{Clasificaciones}
\begin{center}
   	\includegraphics[scale=0.55]{velocidadmenm} 
\end{center}
\tiny{Fuente: Carson \& Kirkby (1972)}
\end{frame}
%%%%%%%%%%%%%%%%%%%%%%%%%%%%%%%%%%%%%%%%%%%%%%%%%%%%%%%%%%%%%%%%%%%%%%%%%
\begin{frame}
\frametitle{Cruden \& Varnes (1996)}
\framesubtitle{Material \& movimiento}
\begin{center}
   	\includegraphics[scale=0.55]{clacificacionmen} 
\end{center}
\tiny{Fuente: Cruden \& Varnes (1996)}
\end{frame}
%%%%%%%%%%%%%%%%%%%%%%%%%%%%%%%%%%%%%%%%%%%%%%%%%%%%%%%%%%%%%%%%%%%%%%%%%
\begin{frame}
\frametitle{Cruden \& Varnes (1996)}
\begin{center}
   	\includegraphics[scale=0.52]{crudenvarnes} 
\end{center}
\tiny{Fuente: Cruden \& Varnes (1996)}
\end{frame}
%%%%%%%%%%%%%%%%%%%%%%%%%%%%%%%%%%%%%%%%%%%%%%%%%%%%%%%%%%%%%%%%%%%%%%%%%
\begin{frame}
\frametitle{Propagación Lateral}
\framesubtitle{\it{Lateral Spreading}}
\begin{itemize}
\item Extensión lateral por fracturas de corte y tensión con una subsidencia general de la masa fracturada en el material subyacente menos consistente.
\item Falla progresiva.
\item Licuefacción o flujo del material menos consistente.
\end{itemize}
\begin{center}
   	\includegraphics[scale=0.52]{lateralspreading} 
\end{center}
\end{frame}
%%%%%%%%%%%%%%%%%%%%%%%%%%%%%%%%%%%%%%%%%%%%%%%%%%%%%%%%%%%%%%%%%%%%%%%%%
\begin{frame}
\frametitle{Volcamiento}
\framesubtitle{\it{Toppling}}
\begin{itemize}
\item Rotación hacia adelante de una masa de suelo o roca sobre un punto o eje bajo el centro de gravedad de la masa desplazada.
\item Gravedad y en ocasiones por agua o hielo en las grietas de la masa.
\item Extremadamente lento a extremadamente rápido
\end{itemize}
\begin{center}
   	\includegraphics[scale=0.52]{toppling} 
\end{center}
\end{frame}
%%%%%%%%%%%%%%%%%%%%%%%%%%%%%%%%%%%%%%%%%%%%%%%%%%%%%%%%%%%%%%%%%%%%%%%%%
\begin{frame}
\frametitle{Volcamiento}
\framesubtitle{\it{Toppling}}
\begin{center}
   	\includegraphics[scale=0.52]{toppling1} 
\end{center}
\end{frame}
%%%%%%%%%%%%%%%%%%%%%%%%%%%%%%%%%%%%%%%%%%%%%%%%%%%%%%%%%%%%%%%%%%%%%%%%%
\begin{frame}
\frametitle{Caida}
\framesubtitle{\it{Fall}}
\begin{itemize}
\small{
\item Inicia con el desprendimiento de suelo o roca de una fuerte pendiente a lo largo de una superficie con un pequeño componente o sin desplazamiento cortante.
\item Principalmente a través del aire por caida, rebote o rodamiento.
\item Muy rápido a extremadamente rápido.
\item Suelos cohesivos o rocas.
\item Márgenes de ríos por erosión o acantilados bajo el ataque de las olas
\item Rodamiento, rebote, caída libre
}
\end{itemize}
\begin{center}
   	\includegraphics[scale=0.4]{caida} 
\end{center}
\end{frame}
%%%%%%%%%%%%%%%%%%%%%%%%%%%%%%%%%%%%%%%%%%%%%%%%%%%%%%%%%%%%%%%%%%%%%%%%%
\begin{frame}
\frametitle{Caida}
\framesubtitle{\it{Fall}}
\begin{center}
   	\includegraphics[scale=0.6]{caida1} 
\end{center}
\end{frame}
%%%%%%%%%%%%%%%%%%%%%%%%%%%%%%%%%%%%%%%%%%%%%%%%%%%%%%%%%%%%%%%%%%%%%%%%%
\begin{frame}
\frametitle{Caida}
\framesubtitle{Depósitos de vertiente: Talus}
\begin{center}
   	\includegraphics[scale=0.52]{talus} 
\end{center}
\end{frame}
%%%%%%%%%%%%%%%%%%%%%%%%%%%%%%%%%%%%%%%%%%%%%%%%%%%%%%%%%%%%%%%%%%%%%%%%%
\begin{frame}
\frametitle{Deslizamiento Rotacional}
\framesubtitle{\it{Rotational Slide}}
\begin{itemize}
\small{
\item Material homogéneo.
\item La superficie de rotura es cóncava y curva.
\item Tiende al equilibrio la masa desplazada.
}
\end{itemize}
\begin{center}
   	\includegraphics[scale=0.8]{rotacional} 
\end{center}
\end{frame}
%%%%%%%%%%%%%%%%%%%%%%%%%%%%%%%%%%%%%%%%%%%%%%%%%%%%%%%%%%%%%%%%%%%%%%%%%
\begin{frame}
\frametitle{Deslizamiento Rotacional}
\framesubtitle{\it{Rotational Slide}}
\begin{center}
   	\includegraphics[scale=0.6]{rotacional1} 
\end{center}
\end{frame}
%%%%%%%%%%%%%%%%%%%%%%%%%%%%%%%%%%%%%%%%%%%%%%%%%%%%%%%%%%%%%%%%%%%%%%%%%
\begin{frame}
\frametitle{Deslizamiento Planar}
\framesubtitle{\it{Traslational Slide}}
\begin{itemize}
\small{
\item La masa se desplaza a lo largo de una superficie planar u ondulada.
\item Mas superficiales a los rotacionales.
\item Sigue discontinuidades como fallas, superficies de depositación o de contacto suelo-roca.
}
\end{itemize}
\begin{center}
   	\includegraphics[scale=0.8]{traslacional} 
\end{center}
\end{frame}
%%%%%%%%%%%%%%%%%%%%%%%%%%%%%%%%%%%%%%%%%%%%%%%%%%%%%%%%%%%%%%%%%%%%%%%%%
\begin{frame}
\frametitle{Deslizamiento Planar}
\framesubtitle{\it{Traslational Slide}}
\begin{center}
   	\includegraphics[scale=0.72]{traslacional1} 
\end{center}
\end{frame}
%%%%%%%%%%%%%%%%%%%%%%%%%%%%%%%%%%%%%%%%%%%%%%%%%%%%%%%%%%%%%%%%%%%%%%%%%
\begin{frame}
\frametitle{Flujos}
\framesubtitle{\it{Flows}}
\begin{itemize}
\small{
\item Movimientos continuos espacialmente en los cuales la superficie cortante es momentanea y no se preserva.
\item Movimientos diferenciales internos se distribuyen a lo largo de la masa desplazada.
\item Gradación desde deslizamientos a flujos dependiendo del contenido de agua, mobilidad y evolución del movimiento.
\item Flujos superficiales \it{Skin flow}, flujos de escombros en laderas abiertas \it{Open-slope debris flow}, flujos canalizados \it{Channelized flows}, Avalanchas de escombros \it{Debris avalanche}, flujos de escombros provenientes de volcanes \it{Lahar}. 
}
\end{itemize}
\begin{center}
   	\includegraphics[scale=0.5]{flows} 
\end{center}
\end{frame}
%%%%%%%%%%%%%%%%%%%%%%%%%%%%%%%%%%%%%%%%%%%%%%%%%%%%%%%%%%%%%%%%%%%%%%%%%
\begin{frame}
\frametitle{Flujos}
\framesubtitle{\it{Flows}}
\begin{center}
   	\includegraphics[scale=0.65]{flujos1} 
\end{center}
\end{frame}
%%%%%%%%%%%%%%%%%%%%%%%%%%%%%%%%%%%%%%%%%%%%%%%%%%%%%%%%%%%%%%%%%%%%%%%%%
\begin{frame}
\frametitle{Flujos}
\framesubtitle{\it{Flows}}
\begin{center}
   	\includegraphics[scale=0.55]{flujos2} 
\end{center}
\end{frame}
%%%%%%%%%%%%%%%%%%%%%%%%%%%%%%%%%%%%%%%%%%%%%%%%%%%%%%%%%%%%%%%%%%%%%%%%%
\begin{frame}
\frametitle{Flujos}
\framesubtitle{\it{Flows}}
\begin{center}
   	\includegraphics[scale=0.48]{flujos3} 
\end{center}
\tiny{\url{BARTALI, Roberto; SAROCCHI, Damiano; NAHMAD-MOLINARI, Yuri  y  RODRIGUEZ-SEDANO, Luis Angel. Estudio de flujos granulares de tipo geológico por medio del simulador multisensor GRANFLOW-SIM. Bol. Soc. Geol. Mex [online]. 2012, vol.64, n.3 [citado  2018-07-09], pp.265-275. Disponible en: <http://www.scielo.org.mx/scielo.php?script=sci_arttext&pid=S1405-33222012000300001&lng=es&nrm=iso>. ISSN 1405-3322}}
\end{frame}
%%%%%%%%%%%%%%%%%%%%%%%%%%%%%%%%%%%%%%%%%%%%%%%%%%%%%%%%%%%%%%%%%%%%%%%%%
\begin{frame}
\frametitle{Flujos}
\framesubtitle{\it{Flows}}
\begin{center}
   	\includegraphics[scale=0.44]{flujoelbarro} 
\end{center}
\end{frame}
%%%%%%%%%%%%%%%%%%%%%%%%%%%%%%%%%%%%%%%%%%%%%%%%%%%%%%%%%%%%%%%%%%%%%%%%%
\begin{frame}
\frametitle{Flujos}
\framesubtitle{\it{Flows}}
\begin{center}
   	\includegraphics[scale=0.35]{flujo4} 
\end{center}
\end{frame}
%%%%%%%%%%%%%%%%%%%%%%%%%%%%%%%%%%%%%%%%%%%%%%%%%%%%%%%%%%%%%%%%%%%%%%%%%
\begin{frame}
\frametitle{Flujos}
\framesubtitle{\it{Flows}}
\begin{center}
   	\includegraphics[scale=0.35]{flujo5} 
\end{center}
\end{frame}
%%%%%%%%%%%%%%%%%%%%%%%%%%%%%%%%%%%%%%%%%%%%%%%%%%%%%%%%%%%%%%%%%%%%%%%%%
\begin{frame}
\frametitle{Entonces este es un...?}
\begin{center}
   	\includegraphics[scale=0.65]{flujo6} 
\end{center}
\end{frame}
%%%%%%%%%%%%%%%%%%%%%%%%%%%%%%%%%%%%%%%%%%%%%%%%%%%%%%%%%%%%%%%%%%%%%%%%%
\begin{frame}
\frametitle{Entonces este es un...?}
\begin{center}
   	\includegraphics[scale=0.55]{flujo7} 
\end{center}
\end{frame}
%%%%%%%%%%%%%%%%%%%%%%%%%%%%%%%%%%%%%%%%%%%%%%%%%%%%%%%%%%%%%%%%%%%%%%%%%
\begin{frame}
\frametitle{Y este...?}
\begin{center}
   	\includegraphics[scale=0.65]{slidecristo} 
\end{center}
\end{frame}
%%%%%%%%%%%%%%%%%%%%%%%%%%%%%%%%%%%%%%%%%%%%%%%%%%%%%%%%%%%%%%%%%%%%%%%%%
\begin{frame}
\frametitle{Reptación}
\framesubtitle{\it{Creeping}}
\begin{center}
   	\includegraphics[scale=0.6]{reptacion}
   	\includegraphics[scale=0.6]{reptacion1}  
\end{center}
\end{frame}
%%%%%%%%%%%%%%%%%%%%%%%%%%%%%%%%%%%%%%%%%%%%%%%%%%%%%%%%%%%%%%%%%%%%%%%%%
\begin{frame}
\normalem
\justifying
\small{
\emph{Creep} es simplemente la deformación que continua bajo esfuerzos constantes (Varnes, 1978).\vfill
La reptación de suelos ha sido considerada como un conjunto de movimientos en masa lentos (continuos y estacionales), causado por temperatura y condiciones de humedad y biota, y balanceado por la topografía, procesos de meteorización de la roca y tasa de producción del suelo.\vfill
La reptación de suelo (\emph{creeping of the surface soil}, Davis (1982)) fue descrito como el resultado de la gravedad, fluctuaciones de la temperatura y la acción de la biota, retrabajado en un fenómeno permanente de dilatación y contracción en climas tropicales.\vfill
Davison (1889) propuso que durante el congelamiento y cambio del nivel freático: (i) expansión del suelo normal a la superficie, pero (ii) contracción vertical, y (iii) la cohesión del suelo previene desplazamientos paralelos a la superficie durante la expansión.\vfill
Los flujos implican movimientos mucho mas rápidos que la reptación, es decir pueden ser perceptibles, mientras que la reptación es imperceptible.\vfill
Movimiento tipo difusivo (\emph{diffusion-like}) movimiento aleatorio de las partículas del suelo resultando en la dispersión desde regiones de alta concentración (densidad) a regiones de baja concentración (Kirkby, 1967).}
\end{frame}
%%%%%%%%%%%%%%%%%%%%%%%%%%%%%%%%%%%%%%%%%%%%%%%%%%%%%%%%%%%%%%%%%%%%%%%%%
\begin{frame}
\frametitle{Reptación}
\framesubtitle{\it{Creeping}}
\begin{center}
   	\includegraphics[scale=0.6]{creep}
\end{center}
\tiny{Fuente: Pawlik \& Samonil (2018)}
\end{frame}
%%%%%%%%%%%%%%%%%%%%%%%%%%%%%%%%%%%%%%%%%%%%%%%%%%%%%%%%%%%%%%%%%%%%%%%%%
\begin{frame}
\frametitle{Reptación}
\framesubtitle{\it{Creeping}}
\begin{center}
   	\includegraphics[scale=0.58]{creep2}
\end{center}
\tiny{Fuente: Pawlik \& Samonil (2018)}
\end{frame}
%%%%%%%%%%%%%%%%%%%%%%%%%%%%%%%%%%%%%%%%%%%%%%%%%%%%%%%%%%%%%%%%%%%%%%%%%
\begin{frame}
\frametitle{Reptación}
\framesubtitle{\it{Creeping}}
\begin{center}
   	\includegraphics[scale=0.58]{creep3}
\end{center}
\tiny{Fuente: Pawlik \& Samonil (2018)}
\end{frame}
%%%%%%%%%%%%%%%%%%%%%%%%%%%%%%%%%%%%%%%%%%%%%%%%%%%%%%%%%%%%%%%%%%%%%%%%%
\begin{frame}
\frametitle{Depósitos de vertiente}
\begin{center}
   	\includegraphics[scale=0.58]{depositos}
\end{center}
\end{frame}
%%%%%%%%%%%%%%%%%%%%%%%%%%%%%%%%%%%%%%%%%%%%%%%%%%%%%%%%%%%%%%%%%%%%%%%%%
\begin{frame}
\frametitle{Proceso continuo espacial y temporalmente}
\begin{center}
   	\includegraphics[scale=0.58]{debrischange}
\end{center}
\end{frame}
%%%%%%%%%%%%%%%%%%%%%%%%%%%%%%%%%%%%%%%%%%%%%%%%%%%%%%%%%%%%%%%%%%%%%%%%%
\begin{frame}
\frametitle{Cárcava o MenM}
\begin{center}
   	\includegraphics[scale=0.58]{carcavamenm}
\end{center}
\end{frame}
%%%%%%%%%%%%%%%%%%%%%%%%%%%%%%%%%%%%%%%%%%%%%%%%%%%%%%%%%%%%%%%%%%%%%%%%%
\begin{frame}
\frametitle{Coluvión}
\begin{itemize}
\item Sedimentos de laderas (Leopold \& Volkel, 2007)
\item Localizados en la base de las laderas (Lapidus, 1990; Soil Survey Staff, 1993; Eggleton, 2001; Millar, 2014)
\item No seleccionado (Marsh \& Kaufman, 2012)
\item Poco transporte (Foucault et al, 2014)
\item Fragmentos angulares (Soil Survey Staff, 1993)
\item Resultado del transporte por gravedad y por flujos no canalizados (Millar , 2014)
\item Transportados bajo la influencia de la gravedad asistidos por agua (Schaetzl \& Thompson, 2015)
\end{itemize}
\vfill
\tiny{\url{https://www.researchgate.net/post/How_do_you_define_colluvium}}
\tiny{Fuente: Miller \& Juilleret (2015)}
\end{frame}
%%%%%%%%%%%%%%%%%%%%%%%%%%%%%%%%%%%%%%%%%%%%%%%%%%%%%%%%%%%%%%%%%%%%%%%%%
\end{document}