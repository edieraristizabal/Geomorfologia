\documentclass{beamer}

\usepackage{beamerthemeCambridgeUS}
\usepackage{textpos}
\usepackage{ragged2e}

\graphicspath{{G:/My Drive/FIGURAS/}}

\title[Ambiente Glacial]{GEOMORFOLOGÍA}
\author[Edier Aristizabal]{Edier V. Aristizábal G.}
\institute{evaristizabalg@unal.edu.co}
\date{Versión:\today}

\addtobeamertemplate{headline}{}{%
	\begin{textblock*}{2mm}(.9\textwidth,0cm)
	\hfill\includegraphics[height=1cm]{un}  
	\end{textblock*}}

\begin{document}
%%%%%%%%%%%%%%%%%%%%%%%%%%%%%%%%%%%%%%%%%%%%%%%%%%%%%%%%%%%%%%%%%%%%%%%%%
\begin{frame}
\titlepage
\centering
	\includegraphics[width=5cm]{unal}\hspace*{4.75cm}~%
   	\includegraphics[width=2cm]{logo3} 
\end{frame}
%%%%%%%%%%%%%%%%%%%%%%%%%%%%%%%%%%%%%%%%%%%%%%%%%%%%%%%%%%%%%%%%%%%%%%%%%
\begin{frame}
\frametitle{Alta Montaña en Colombia}
\begin{center}
\includegraphics[scale=0.52]{altamontaña}
\end{center}
\end{frame}
%%%%%%%%%%%%%%%%%%%%%%%%%%%%%%%%%%%%%%%%%%%%%%%%%%%%%%%%%%%%%%%%%%%%%%%
\begin{frame}
\frametitle{Definición de Glaciar}
\begin{itemize}
\small{
\item Gruesa masa de hielo
\item Se origina en la superficie terrestre por acumulación, compactación y recristalización de la nieve
\item Con evidencias de flujo en el pasado o en la actualidad,
Su existencia es posible cuando la precipitación anual de nieve supera la evaporada en verano
\item La mayoría se encuentra en zonas cercanas a los polos, aunque existen en otras zonas montañosas.
\item El proceso del crecimiento y establecimiento del glaciar se llama glaciación.
}
\end{itemize}
\begin{center}
\includegraphics[scale=0.52]{glaciar}
\end{center}
\end{frame}
%%%%%%%%%%%%%%%%%%%%%%%%%%%%%%%%%%%%%%%%%%%%%%%%%%%%%%%%%%%%%%%%%%%%%%%
\begin{frame}
\frametitle{Tipos de Glaciar}
\framesubtitle{Altitud alta (alpinos) \& Latitud alta (ice sheet, continental glacier)}
\begin{center}
\includegraphics[scale=0.48]{tiposglaciar}
\end{center}
\end{frame}
%%%%%%%%%%%%%%%%%%%%%%%%%%%%%%%%%%%%%%%%%%%%%%%%%%%%%%%%%%%%%%%%%%%%%%%
\begin{frame}
\frametitle{Glaciar continental de casquete}
\justifying
\small{Los glaciares más grandes son los glaciares continentales de casquete. Enormes masas de hielo que no son afectadas por el paisaje y se extienden por toda la superficie, excepto en los márgenes, donde su espesor es más delgado.}
\begin{columns}
	\begin{column}{.65\linewidth}
	\begin{center}
	\includegraphics[scale=0.45]{antarctica}
	\end{center}
	\end{column}
	\begin{column}{.35\linewidth}
	\begin{center}
\tiny{A RADARSAT, Map of Antarctica, November 16, 1999 
Antarctica covers nearly 9\% of the Earth’s land, and is 25\% bigger than Europe, making it the fifth largest continent. About 99\% of Antarctica is covered with a vast ice sheet. It is the largest single mass of ice on Earth. The ice sheet averages 2,450 metres deep and holds about 70\% of the world’s fresh water. With such a thick layer of ice, Antarctica is the highest of all the continents. The average altitude is about 2,300 metres above sea level, although in places, the bottom of the ice can be as much as 3,000m below sea level. If they weren’t filled with ice, large parts of Antarctica would be under the sea. Vinson Massif is Antarctica’s highest point, rising to a height of 4,897 metres.}
	\end{center}
	\end{column}
\end{columns}
\end{frame}
%%%%%%%%%%%%%%%%%%%%%%%%%%%%%%%%%%%%%%%%%%%%%%%%%%%%%%%%%%%%%%%%%%%%%%%
\begin{frame}
\frametitle{Glaciar continental de casquete}
\justifying
\small{es una masa de hielo que cubre un área menor que 50.000 $km^2$ en las regiones polares de la Tierra. Las masas de hielo mayores son designadas como Indlandsi (hielo interior); su masa es menor que la presente en los glaciares continentales.}
\begin{columns}
	\begin{column}{.2\linewidth}
	\begin{center}
	\includegraphics[scale=0.4]{mapamundi}
	\end{center}
	\end{column}
	\begin{column}{.8\linewidth}
	\begin{center}
	\includegraphics[scale=0.55]{groenlandia}
	\end{center}
	\end{column}
\end{columns}
\end{frame}
%%%%%%%%%%%%%%%%%%%%%%%%%%%%%%%%%%%%%%%%%%%%%%%%%%%%%%%%%%%%%%%%%%%%%%%
\begin{frame}
\frametitle{Glaciar continental de casquete}
\begin{center}
\includegraphics[scale=0.48]{icesheet}
\end{center}
\end{frame}
%%%%%%%%%%%%%%%%%%%%%%%%%%%%%%%%%%%%%%%%%%%%%%%%%%%%%%%%%%%%%%%%%%%%%%%
\begin{frame}
\frametitle{Glaciar Alpino}
\small{Esta clase incluye a los glaciares más pequeños, los cuales se caracterizan por estar confinados en los valles montañosos, razón por la que se los denomina glaciares de valle o alpinos o de montaña.}
\begin{center}
\includegraphics[scale=0.48]{glaciaralpino}
\end{center}
\end{frame}
%%%%%%%%%%%%%%%%%%%%%%%%%%%%%%%%%%%%%%%%%%%%%%%%%%%%%%%%%%%%%%%%%%%%%%%
\begin{frame}
\frametitle{Otros tipos de Glaciar}
\framesubtitle{Glaciar de Piedemonte}
\begin{center}
\includegraphics[scale=0.45]{glaciarpiedemonte}
\end{center}
\end{frame}
%%%%%%%%%%%%%%%%%%%%%%%%%%%%%%%%%%%%%%%%%%%%%%%%%%%%%%%%%%%%%%%%%%%%%%%
\begin{frame}
\frametitle{Otros tipos de Glaciar}
\framesubtitle{Glaciar de Meseta}
\begin{center}
\includegraphics[scale=1]{glaciarmeseta}
\end{center}
\end{frame}
%%%%%%%%%%%%%%%%%%%%%%%%%%%%%%%%%%%%%%%%%%%%%%%%%%%%%%%%%%%%%%%%%%%%%%%
\begin{frame}
\frametitle{Otros tipos de Glaciar}
\framesubtitle{Glaciar outlet, exutorio, efluentes}
\begin{center}
\includegraphics[scale=0.92]{outletglacier}
\end{center}
\end{frame}
%%%%%%%%%%%%%%%%%%%%%%%%%%%%%%%%%%%%%%%%%%%%%%%%%%%%%%%%%%%%%%%%%%%%%%%
\begin{frame}
\begin{center}
\includegraphics[scale=0.45]{locglaciar}
\end{center}
\end{frame}
%%%%%%%%%%%%%%%%%%%%%%%%%%%%%%%%%%%%%%%%%%%%%%%%%%%%%%%%%%%%%%%%%%%%%%%
\begin{frame}
\frametitle{Glaciares en Colombia}
\begin{center}
\includegraphics[scale=0.45]{glaciarcolombia}
\end{center}
\end{frame}
%%%%%%%%%%%%%%%%%%%%%%%%%%%%%%%%%%%%%%%%%%%%%%%%%%%%%%%%%%%%%%%%%%%%%%%%%
\begin{frame}
\frametitle{Glaciares Extintos en Colombia}
\framesubtitle{374 $km^2$ $\rightarrow$ 37 $km^2$}
\begin{center}
\includegraphics[scale=0.40]{glaciaresextintos}
\end{center}
\tiny{Glaciares colombianos extintos en el siglo XX. Fuente: Florez, 2002. Elaboración: IDEAM, 2018.}
\end{frame}
%%%%%%%%%%%%%%%%%%%%%%%%%%%%%%%%%%%%%%%%%%%%%%%%%%%%%%%%%%%%%%%%%%%%%%%%%
\begin{frame}
\frametitle{Glaciares Extintos en Colombia}
\framesubtitle{374 $km^2$ $\rightarrow$ 37 $km^2$}
\begin{center}
\includegraphics[scale=0.45]{evolucionglaciar1}
\end{center}
\end{frame}
%%%%%%%%%%%%%%%%%%%%%%%%%%%%%%%%%%%%%%%%%%%%%%%%%%%%%%%%%%%%%%%%%%%%%%%%%
\begin{frame}
\frametitle{Origen de los glaciares}
\begin{center}
\includegraphics[scale=0.45]{pleistoceno}
\end{center}
\end{frame}
%%%%%%%%%%%%%%%%%%%%%%%%%%%%%%%%%%%%%%%%%%%%%%%%%%%%%%%%%%%%%%%%%%%%%%%%%
\begin{frame}
\frametitle{Origen de los glaciares}
\begin{center}
\includegraphics[scale=0.48]{glaciacionestiempo}
\end{center}
\tiny{Fuente: \url{http://intheplaygroundofgiants.com/geology-of-central-oregon/the-geology-of-glaciers-and-glaciation/}}
\end{frame}
%%%%%%%%%%%%%%%%%%%%%%%%%%%%%%%%%%%%%%%%%%%%%%%%%%%%%%%%%%%%%%%%%%%%%%%%%
\begin{frame}
\frametitle{Cambio Climático o Variabilidad Climática?}
\begin{center}
\includegraphics[scale=0.48]{variabilidadclimatica}
\end{center}
\end{frame}
%%%%%%%%%%%%%%%%%%%%%%%%%%%%%%%%%%%%%%%%%%%%%%%%%%%%%%%%%%%%%%%%%%%%%%%%%
\begin{frame}
\frametitle{Origen de los glaciares}
\begin{center}
\includegraphics[scale=0.48]{cambiostiempo}
\end{center}
\end{frame}
%%%%%%%%%%%%%%%%%%%%%%%%%%%%%%%%%%%%%%%%%%%%%%%%%%%%%%%%%%%%%%%%%%%%%%%%%
\begin{frame}
\frametitle{Origen de los glaciares}
\justifying
\small{
\textbf{Excentricidad de la órbita}, basada en lo estirada que está de la elipse. Si la órbita de la Tierra es más elíptica la excentricidad es mayor y al contrario si es más circular. La excentricidad varía entre sus valores extremos cada 100,000 años, y esta variación puede suponer entre un 1\% y un 11\% de diferencia en la cantidad de radiación solar que recibe la Tierra entre el afelio y el perihelio. En la actualidad, entre el afelio y el perihelio la cantidad de radiación que llega a la Tierra cambia un 6\%.\\
\textbf{Oblicuidad}: cambios en el ángulo del eje de rotación de la tierra (+ ó – inclinación). La inclinación oscila entre 21,6º y 24.5º cada 40,000 años. Actualmente está en 23,5º. Este fenómeno es el responsable de las estaciones. Aunque no cambia la cantidad de radiación que recibe la Tierra, sí varía su distribución sobre la superficie.\\
\textbf{Precesión}: giro del eje de rotación, en sentido contrario a la rotación, cada 26,000 años. Se debe al achatamiento de la esfera terrestre. Su efecto sobre el clima es consecuencia de la modificación de la posición relativa de los solsticios y los equinoccios respecto al afelio y al perihelio. En la actualidad el solsticio de verano coincide con el afelio en el hemisferio norte. Dentro de 6,000 años el afelio coincidirá con el equinoccio de primavera, y dentro de unos 12,000 años el solsticio de verano coincidirá con el perihelio.}
\end{frame}
%%%%%%%%%%%%%%%%%%%%%%%%%%%%%%%%%%%%%%%%%%%%%%%%%%%%%%%%%%%%%%%%%%%%%%%%%
\begin{frame}
\frametitle{Origen de los glaciares}
\begin{center}
\includegraphics[scale=0.48]{milankovitch2}
\end{center}
\end{frame}
%%%%%%%%%%%%%%%%%%%%%%%%%%%%%%%%%%%%%%%%%%%%%%%%%%%%%%%%%%%%%%%%%%%%%%%%%
\begin{frame}
\frametitle{Origen de los glaciares}
\small{Las variaciones orbitales son las causantes de los períodos glaciales e interglaciales producidos durante este último período (Holoceno).\\ 
\textbf{Glaciaciones}: periodos de alta excentricidad, baja inclinación y una distancia grande Tierra -Sol, en verano (Hemisferio norte). El resultado sería un débil contraste estacional.\\
\textbf{Interglaciares}: baja excentricidad, gran inclinación, y distancia Tierra-Sol en verano baja. Resultando en estaciones contrastadas.}
\begin{center}
\includegraphics[scale=0.48]{milankovitch}
\end{center}
\end{frame}
%%%%%%%%%%%%%%%%%%%%%%%%%%%%%%%%%%%%%%%%%%%%%%%%%%%%%%%%%%%%%%%%%%%%%%%%%
\begin{frame}
\frametitle{Glaciar o Glacial?}
Glaciar $\rightarrow$ Sustantivo\\
Cuerpos de hielo
\vfill 
Glacial $\rightarrow$ Sustantivo \& Adjetivo\\
Periodos mas fríos en la historia geológica del planeta
Geoformas y procesos que producen glaciares (geoformas glaciales)
\end{frame}
%%%%%%%%%%%%%%%%%%%%%%%%%%%%%%%%%%%%%%%%%%%%%%%%%%%%%%%%%%%%%%%%%%%%%%%%%
\begin{frame}
\frametitle{Cómo se forma un glaciar?}
\begin{center}
\includegraphics[scale=0.45]{glaciar1}
\end{center}
\end{frame}
%%%%%%%%%%%%%%%%%%%%%%%%%%%%%%%%%%%%%%%%%%%%%%%%%%%%%%%%%%%%%%%%%%%%%%%%%
\begin{frame}
\frametitle{Dinámica glaciar}
\begin{center}
\includegraphics[scale=0.45]{dinamicaglaciar}
\end{center}
\end{frame}
%%%%%%%%%%%%%%%%%%%%%%%%%%%%%%%%%%%%%%%%%%%%%%%%%%%%%%%%%%%%%%%%%%%%%%%%%
\begin{frame}
\frametitle{\emph{Snowline}}
\begin{center}
\includegraphics[scale=0.49]{snowline}
\end{center}
\end{frame}
%%%%%%%%%%%%%%%%%%%%%%%%%%%%%%%%%%%%%%%%%%%%%%%%%%%%%%%%%%%%%%%%%%%%%%%%%
\begin{frame}
\frametitle{Circo Glacial}
\small{Geoforma cóncava erodada como depresiones cerca al tope de las montañas donde la nieve se acumula y forma la parte alta del los glaciales alpinos. Corresponden a las etapas tempranas de los glaciales y son la ultima parte en fundirse del glacial en su retroceso.}
\begin{center}
\includegraphics[scale=0.49]{circoglacial}
\end{center}
\end{frame}
%%%%%%%%%%%%%%%%%%%%%%%%%%%%%%%%%%%%%%%%%%%%%%%%%%%%%%%%%%%%%%%%%%%%%%%%%
\begin{frame}
\frametitle{Circo Glacial}
\justifying
\small{Geoforma cóncava erodada como depresiones cerca al tope de las montañas donde la nieve se acumula y forma la parte alta del los glaciales alpinos. Corresponden a las etapas tempranas de los glaciales y son la ultima parte en fundirse del glacial en su retroceso.}
\begin{center}
\includegraphics[scale=0.49]{circoglacial}
\end{center}
\end{frame}
%%%%%%%%%%%%%%%%%%%%%%%%%%%%%%%%%%%%%%%%%%%%%%%%%%%%%%%%%%%%%%%%%%%%%%%%%
\begin{frame}
\frametitle{Circo Glacial}
\begin{figure}
\begin{center}
\includegraphics[scale=0.45]{circoglacial2}
\end{center}
\caption{Imagen del circo glaciar Tjønnholstinden, en Noruega (foto de Tore Røraas / Marianne Gjørv)}
\end{figure}
\end{frame}
%%%%%%%%%%%%%%%%%%%%%%%%%%%%%%%%%%%%%%%%%%%%%%%%%%%%%%%%%%%%%%%%%%%%%%%%%
\begin{frame}
\frametitle{Circo Glacial}
\begin{figure}
\begin{center}
\includegraphics[scale=0.49]{circoglacial3}
\end{center}
\caption{Disko Island in Greenland}
\end{figure}
\end{frame}
%%%%%%%%%%%%%%%%%%%%%%%%%%%%%%%%%%%%%%%%%%%%%%%%%%%%%%%%%%%%%%%%%%%%%%%%%
\begin{frame}
\frametitle{Circo Glacial}
\begin{figure}
\begin{center}
\includegraphics[scale=0.49]{circoglacial4}
\end{center}
\caption{The cirque of Cwm Cau on the peak of Cadair Idris, Snowdonia National Park, Wales. Photo M. J. Hambrey.}
\end{figure}
\end{frame}
%%%%%%%%%%%%%%%%%%%%%%%%%%%%%%%%%%%%%%%%%%%%%%%%%%%%%%%%%%%%%%%%%%%%%%%%%
\begin{frame}
\frametitle{Circo Glacial \& Tarn}
\begin{figure}
\begin{center}
\includegraphics[scale=0.45]{circoglacial5}
\end{center}
\caption{Two tarns on the Glyderau range, Snowdonia, North Wales, both occupying cirques. The tarn on the left is Llyn Bochllwyd and that in the centre is Llyn Llyn Idwal ("llyn" is Welsh for "lake"). Photo M. J. Hambrey.}
\end{figure}
\end{frame}
%%%%%%%%%%%%%%%%%%%%%%%%%%%%%%%%%%%%%%%%%%%%%%%%%%%%%%%%%%%%%%%%%%%%%%%%%
\begin{frame}
\frametitle{Circo Glacial \& Tarn}
\begin{figure}
\begin{center}
\includegraphics[scale=0.49]{circoglacial6}
\end{center}
\end{figure}
\end{frame}
%%%%%%%%%%%%%%%%%%%%%%%%%%%%%%%%%%%%%%%%%%%%%%%%%%%%%%%%%%%%%%%%%%%%%%%%%
\begin{frame}
\frametitle{Lagos Paternoster}
\small{Son una serie de lagos glaciales conectados creados por erosión diferencial o por represamiento por morrenas}
\begin{figure}
\begin{center}
\includegraphics[scale=0.49]{paternoster}
\end{center}
\end{figure}
\end{frame}
%%%%%%%%%%%%%%%%%%%%%%%%%%%%%%%%%%%%%%%%%%%%%%%%%%%%%%%%%%%%%%%%%%%%%%%%%
\begin{frame}
\frametitle{Lagos Paternoster}
\begin{figure}
\begin{center}
\includegraphics[scale=0.49]{paternoster1}
\end{center}
\end{figure}
\end{frame}
%%%%%%%%%%%%%%%%%%%%%%%%%%%%%%%%%%%%%%%%%%%%%%%%%%%%%%%%%%%%%%%%%%%%%%%%%
\begin{frame}
\frametitle{Lagos Paternoster}
\begin{figure}
\begin{center}
\includegraphics[scale=0.49]{paternoster2}
\end{center}
\caption{Alderson / Carthew lakes / summit area, depending on who you ask. Waterton Lakes National Park, Alberta.}
\end{figure}
\end{frame}
%%%%%%%%%%%%%%%%%%%%%%%%%%%%%%%%%%%%%%%%%%%%%%%%%%%%%%%%%%%%%%%%%%%%%%%%%
\begin{frame}
\frametitle{Horn, Arete, \& Col}
\begin{figure}
\begin{center}
\includegraphics[scale=0.6]{horn}
\end{center}
\end{figure}
\end{frame}
%%%%%%%%%%%%%%%%%%%%%%%%%%%%%%%%%%%%%%%%%%%%%%%%%%%%%%%%%%%%%%%%%%%%%%%%%
\begin{frame}
\frametitle{Arete}
\begin{figure}
\begin{center}
\includegraphics[scale=0.6]{aretes}
\end{center}
\caption{Bergschrunds in upper Franz Josef Glacier with the 3000 m-high Main Divide}
\end{figure}
\end{frame}
%%%%%%%%%%%%%%%%%%%%%%%%%%%%%%%%%%%%%%%%%%%%%%%%%%%%%%%%%%%%%%%%%%%%%%%%%
\begin{frame}
\frametitle{Arete}
\begin{figure}
\begin{center}
\includegraphics[scale=0.5]{aretes1}
\end{center}
\end{figure}
\end{frame}
%%%%%%%%%%%%%%%%%%%%%%%%%%%%%%%%%%%%%%%%%%%%%%%%%%%%%%%%%%%%%%%%%%%%%%%%%
\begin{frame}
\frametitle{Cols}
\small{Crestas en forma de silla de montar formada por dos circos en lados opuestos que erodaron el arete que los separaba.}
\begin{figure}
\begin{center}
\includegraphics[scale=0.5]{cols}
\end{center}
\end{figure}
\end{frame}
%%%%%%%%%%%%%%%%%%%%%%%%%%%%%%%%%%%%%%%%%%%%%%%%%%%%%%%%%%%%%%%%%%%%%%%%%
\begin{frame}
\frametitle{Cols}
\begin{figure}
\begin{center}
\includegraphics[scale=0.47]{cols1}
\end{center}
\end{figure}
\end{frame}
%%%%%%%%%%%%%%%%%%%%%%%%%%%%%%%%%%%%%%%%%%%%%%%%%%%%%%%%%%%%%%%%%%%%%%%%%
\begin{frame}
\frametitle{Horns}
\small{Pico piramidal formado por la conjunción y erosión de circos y valles glaciales. }
\begin{figure}
\begin{center}
\includegraphics[scale=0.47]{horns}
\end{center}
\end{figure}
\end{frame}
%%%%%%%%%%%%%%%%%%%%%%%%%%%%%%%%%%%%%%%%%%%%%%%%%%%%%%%%%%%%%%%%%%%%%%%%%
\begin{frame}
\frametitle{Horns}
\begin{figure}
\begin{center}
\includegraphics[scale=0.47]{horns1}
\end{center}
\caption{Logan Pass, in Glacier National Park}
\end{figure}
\end{frame}
%%%%%%%%%%%%%%%%%%%%%%%%%%%%%%%%%%%%%%%%%%%%%%%%%%%%%%%%%%%%%%%%%%%%%%%%%
\begin{frame}
\frametitle{Horns}
\begin{figure}
\begin{center}
\includegraphics[scale=0.52]{horns2}
\end{center}
\end{figure}
\end{frame}
%%%%%%%%%%%%%%%%%%%%%%%%%%%%%%%%%%%%%%%%%%%%%%%%%%%%%%%%%%%%%%%%%%%%%%%%%
\begin{frame}
\frametitle{Zona de ablación}
\begin{figure}
\begin{center}
\includegraphics[scale=0.48]{ablacion}
\end{center}
\end{figure}
\end{frame}
%%%%%%%%%%%%%%%%%%%%%%%%%%%%%%%%%%%%%%%%%%%%%%%%%%%%%%%%%%%%%%%%%%%%%%%%%
\begin{frame}
\frametitle{Zona de ablación}
\begin{figure}
\begin{center}
\includegraphics[scale=0.48]{ablacion1}
\end{center}
\end{figure}
\end{frame}
%%%%%%%%%%%%%%%%%%%%%%%%%%%%%%%%%%%%%%%%%%%%%%%%%%%%%%%%%%%%%%%%%%%%%%%%%
\begin{frame}
\frametitle{Morrenas}
\begin{figure}
\begin{center}
\includegraphics[scale=0.41]{morrenas}
\end{center}
\caption{Alaska Wrangell St Elias national park Barnard glacier shows good example of medial moraines}
\end{figure}
\end{frame}
%%%%%%%%%%%%%%%%%%%%%%%%%%%%%%%%%%%%%%%%%%%%%%%%%%%%%%%%%%%%%%%%%%%%%%%%%
\begin{frame}
\frametitle{Valles en "U"}
\begin{figure}
\begin{center}
\includegraphics[scale=0.45]{vallesu}
\end{center}
\caption{El Valle de Ordesa es uno de los valles glaciares del geoparque de Sobrarbe, con las típicas paredes verticales y la base más o menos plana de este tipo de valles con forma de U (fuente: luzmediterranea.wordpress.com)}
\end{figure}
\end{frame}
%%%%%%%%%%%%%%%%%%%%%%%%%%%%%%%%%%%%%%%%%%%%%%%%%%%%%%%%%%%%%%%%%%%%%%%%%
\begin{frame}
\frametitle{Valles en "U"}
\begin{figure}
\begin{center}
\includegraphics[scale=0.45]{vallesu1}
\end{center}
\end{figure}
\end{frame}
%%%%%%%%%%%%%%%%%%%%%%%%%%%%%%%%%%%%%%%%%%%%%%%%%%%%%%%%%%%%%%%%%%%%%%%%%
\begin{frame}
\frametitle{Valles en "U"}
\begin{figure}
\begin{center}
\includegraphics[scale=0.48]{vallesu2}
\end{center}
\end{figure}
\end{frame}
%%%%%%%%%%%%%%%%%%%%%%%%%%%%%%%%%%%%%%%%%%%%%%%%%%%%%%%%%%%%%%%%%%%%%%%%%
\begin{frame}
\frametitle{Valles en "U"}
\begin{figure}
\begin{center}
\includegraphics[scale=0.45]{vallesu3}
\end{center}
\end{figure}
\end{frame}
%%%%%%%%%%%%%%%%%%%%%%%%%%%%%%%%%%%%%%%%%%%%%%%%%%%%%%%%%%%%%%%%%%%%%%%%%
\begin{frame}
\frametitle{Valles Colgados}
\begin{figure}
\begin{center}
\includegraphics[scale=0.54]{vallescolgados}
\end{center}
\end{figure}
\end{frame}
%%%%%%%%%%%%%%%%%%%%%%%%%%%%%%%%%%%%%%%%%%%%%%%%%%%%%%%%%%%%%%%%%%%%%%%%%
\begin{frame}
\frametitle{Valles Colgados}
\begin{figure}
\begin{center}
\includegraphics[scale=0.50]{vallescolgados1}
\end{center}
\end{figure}
\end{frame}
%%%%%%%%%%%%%%%%%%%%%%%%%%%%%%%%%%%%%%%%%%%%%%%%%%%%%%%%%%%%%%%%%%%%%%%%%
\begin{frame}
\frametitle{Valles Colgados}
\begin{figure}
\begin{center}
\includegraphics[scale=0.52]{vallescolgados2}
\end{center}
\end{figure}
\end{frame}
%%%%%%%%%%%%%%%%%%%%%%%%%%%%%%%%%%%%%%%%%%%%%%%%%%%%%%%%%%%%%%%%%%%%%%%%%
\begin{frame}
\frametitle{Fjords}
\small{Es un profundo, estrecho y alongado valle glacial parcialmente inundado por el mar. Pueden ser el resultado de glaciación activa o post glaciación, dependiendo el nivel el mar.}
\begin{figure}
\begin{center}
\includegraphics[scale=0.52]{fjord}
\end{center}
\end{figure}
\end{frame}
%%%%%%%%%%%%%%%%%%%%%%%%%%%%%%%%%%%%%%%%%%%%%%%%%%%%%%%%%%%%%%%%%%%%%%%%%
\begin{frame}
\frametitle{Fjords}
\begin{figure}
\begin{center}
\includegraphics[scale=0.52]{fjord1}
\end{center}
\end{figure}
\end{frame}
%%%%%%%%%%%%%%%%%%%%%%%%%%%%%%%%%%%%%%%%%%%%%%%%%%%%%%%%%%%%%%%%%%%%%%%%%
\begin{frame}
\frametitle{Fjords}
\begin{figure}
\begin{center}
\includegraphics[scale=0.50]{fjord2}
\end{center}
\end{figure}
\end{frame}
%%%%%%%%%%%%%%%%%%%%%%%%%%%%%%%%%%%%%%%%%%%%%%%%%%%%%%%%%%%%%%%%%%%%%%%%%
\begin{frame}
\frametitle{Morrenas}
\begin{figure}
\begin{center}
\includegraphics[scale=0.46]{morrenas1}
\end{center}
\end{figure}
\end{frame}
%%%%%%%%%%%%%%%%%%%%%%%%%%%%%%%%%%%%%%%%%%%%%%%%%%%%%%%%%%%%%%%%%%%%%%%%%
\begin{frame}
\frametitle{Morrenas}
\justifying
\small{
El hielo que se va desplazando formando la lengua glaciar a su paso arranca materiales de dicho terreno actuando como una lija.\vfill
Los \textbf{Tills} son los sedimentos arrastrados por el glaciar. Cuando se acumulan, se compactan mediante el proceso de diagénesis, y forman las tillitas.\vfill 
En ocasiones el till se presenta dando formas de relieve características que se conocen con el nombre de morrena. Las Morrenas glaciares son la forma de depósito glaciar más importante y conocida.\vfill
Las \textbf{Morrenas} son acumulaciones de material transportado por el glaciar que aparecen en las zonas donde el hielo desaparece. Según dónde encontremos este tipo de derrubios de tamaños muy diversos se habla de:
\begin{itemize}
\item morrenas terminales (en el frente del glaciar), 
\item morrenas laterales (a los lados de la lengua de hielo),
\item morrena central; es la que resulta al juntarse dos morrenas laterales, y forman una única.
\item morrenas de fondo (en la base del valle glaciar). 
\end{itemize}
}
\end{frame}
%%%%%%%%%%%%%%%%%%%%%%%%%%%%%%%%%%%%%%%%%%%%%%%%%%%%%%%%%%%%%%%%%%%%%%%%%
\begin{frame}
\frametitle{Morrenas}
\begin{figure}
\begin{center}
\includegraphics[scale=0.54]{morrenas2}
\end{center}
\end{figure}
\end{frame}
%%%%%%%%%%%%%%%%%%%%%%%%%%%%%%%%%%%%%%%%%%%%%%%%%%%%%%%%%%%%%%%%%%%%%%%%%
\begin{frame}
\frametitle{Morrenas}
\begin{figure}
\begin{center}
\includegraphics[scale=0.47]{morrenas3}
\end{center}
\end{figure}
\end{frame}
%%%%%%%%%%%%%%%%%%%%%%%%%%%%%%%%%%%%%%%%%%%%%%%%%%%%%%%%%%%%%%%%%%%%%%%%%
\begin{frame}
\frametitle{Geoformas asociadas a retroceso glaciar}
\begin{figure}
\begin{center}
\includegraphics[scale=0.47]{glaciarretro}
\end{center}
\end{figure}
\end{frame}
%%%%%%%%%%%%%%%%%%%%%%%%%%%%%%%%%%%%%%%%%%%%%%%%%%%%%%%%%%%%%%%%%%%%%%%%%
\begin{frame}
\frametitle{Geoformas asociadas a retroceso glaciar}
\begin{figure}
\begin{center}
\includegraphics[scale=0.47]{glaciarretro1}
\end{center}
\end{figure}
\end{frame}
%%%%%%%%%%%%%%%%%%%%%%%%%%%%%%%%%%%%%%%%%%%%%%%%%%%%%%%%%%%%%%%%%%%%%%%%%
\begin{frame}
\frametitle{Kettles}
\begin{figure}
\begin{center}
\includegraphics[scale=0.45]{kettles}\vfill
\includegraphics[scale=0.45]{kettles1}
\end{center}
\end{figure}
\end{frame}
%%%%%%%%%%%%%%%%%%%%%%%%%%%%%%%%%%%%%%%%%%%%%%%%%%%%%%%%%%%%%%%%%%%%%%%%%
\begin{frame}
\frametitle{Kettles}
\begin{figure}
\begin{center}
\includegraphics[scale=0.54]{kettles2}
\end{center}
\end{figure}
\end{frame}
%%%%%%%%%%%%%%%%%%%%%%%%%%%%%%%%%%%%%%%%%%%%%%%%%%%%%%%%%%%%%%%%%%%%%%%%%
\begin{frame}
\frametitle{Kames \& Eskers}
\begin{figure}
\begin{center}
\includegraphics[scale=0.54]{kames}
\end{center}
\end{figure}
\end{frame}
%%%%%%%%%%%%%%%%%%%%%%%%%%%%%%%%%%%%%%%%%%%%%%%%%%%%%%%%%%%%%%%%%%%%%%%%%
\begin{frame}
\frametitle{Kames \& Eskers}
\begin{figure}
\begin{center}
\includegraphics[scale=0.6]{kames1}
\end{center}
\end{figure}
\end{frame}
%%%%%%%%%%%%%%%%%%%%%%%%%%%%%%%%%%%%%%%%%%%%%%%%%%%%%%%%%%%%%%%%%%%%%%%%%
\begin{frame}
\frametitle{Kames}
\begin{figure}
\begin{center}
\includegraphics[scale=0.51]{kames2}
\end{center}
\end{figure}
\end{frame}
%%%%%%%%%%%%%%%%%%%%%%%%%%%%%%%%%%%%%%%%%%%%%%%%%%%%%%%%%%%%%%%%%%%%%%%%%
\begin{frame}
\frametitle{Kames}
\begin{figure}
\begin{center}
\includegraphics[scale=0.54]{kames3}
\end{center}
\end{figure}
\end{frame}
%%%%%%%%%%%%%%%%%%%%%%%%%%%%%%%%%%%%%%%%%%%%%%%%%%%%%%%%%%%%%%%%%%%%%%%%%
\begin{frame}
\frametitle{Eskers}
\begin{figure}
\begin{center}
\includegraphics[scale=0.54]{eskers}
\end{center}
\end{figure}
\end{frame}
%%%%%%%%%%%%%%%%%%%%%%%%%%%%%%%%%%%%%%%%%%%%%%%%%%%%%%%%%%%%%%%%%%%%%%%%%
\begin{frame}
\frametitle{Eskers}
\begin{figure}
\begin{center}
\includegraphics[scale=0.51]{eskers1}
\end{center}
\end{figure}
\end{frame}
%%%%%%%%%%%%%%%%%%%%%%%%%%%%%%%%%%%%%%%%%%%%%%%%%%%%%%%%%%%%%%%%%%%%%%%%%
\begin{frame}
\frametitle{Eskers}
\begin{figure}
\begin{center}
\includegraphics[scale=0.48]{eskers2}
\end{center}
\end{figure}
\end{frame}
%%%%%%%%%%%%%%%%%%%%%%%%%%%%%%%%%%%%%%%%%%%%%%%%%%%%%%%%%%%%%%%%%%%%%%%%%
\begin{frame}
\frametitle{Drumlins}
\begin{figure}
\begin{center}
\includegraphics[scale=0.48]{dumlins}
\end{center}
\end{figure}
\end{frame}
%%%%%%%%%%%%%%%%%%%%%%%%%%%%%%%%%%%%%%%%%%%%%%%%%%%%%%%%%%%%%%%%%%%%%%%%%
\begin{frame}
\frametitle{Drumlins}
\begin{figure}
\begin{center}
\includegraphics[scale=0.48]{drumlins}
\end{center}
\end{figure}
\end{frame}
%%%%%%%%%%%%%%%%%%%%%%%%%%%%%%%%%%%%%%%%%%%%%%%%%%%%%%%%%%%%%%%%%%%%%%%%%
\begin{frame}
\frametitle{Drumlins}
\begin{figure}
\begin{center}
\includegraphics[scale=0.48]{drumlins1}
\end{center}
\end{figure}
\end{frame}
%%%%%%%%%%%%%%%%%%%%%%%%%%%%%%%%%%%%%%%%%%%%%%%%%%%%%%%%%%%%%%%%%%%%%%%%%
\begin{frame}
\frametitle{Rocas Aborregadas}
\begin{figure}
\begin{center}
\includegraphics[scale=0.48]{aborregadas}
\end{center}
\end{figure}
\end{frame}
%%%%%%%%%%%%%%%%%%%%%%%%%%%%%%%%%%%%%%%%%%%%%%%%%%%%%%%%%%%%%%%%%%%%%%%%%
\begin{frame}
\frametitle{Estrias}
\begin{figure}
\begin{center}
\includegraphics[scale=0.48]{estrias}
\end{center}
\end{figure}
\end{frame}
%%%%%%%%%%%%%%%%%%%%%%%%%%%%%%%%%%%%%%%%%%%%%%%%%%%%%%%%%%%%%%%%%%%%%%%%%
\begin{frame}
\frametitle{Estrias}
\begin{figure}
\begin{center}
\includegraphics[scale=0.48]{estrias1}
\end{center}
\end{figure}
\end{frame}
%%%%%%%%%%%%%%%%%%%%%%%%%%%%%%%%%%%%%%%%%%%%%%%%%%%%%%%%%%%%%%%%%%%%%%%%%
\begin{frame}
\frametitle{Bloques Erraticos}
\begin{figure}
\begin{center}
\includegraphics[scale=0.48]{bloqueserraticos}
\end{center}
\end{figure}
\end{frame}
%%%%%%%%%%%%%%%%%%%%%%%%%%%%%%%%%%%%%%%%%%%%%%%%%%%%%%%%%%%%%%%%%%%%%%%%%
\begin{frame}
\frametitle{\emph{Outwash}}
\begin{figure}
\begin{center}
\includegraphics[scale=0.48]{outwash}
\end{center}
\end{figure}
\end{frame}
%%%%%%%%%%%%%%%%%%%%%%%%%%%%%%%%%%%%%%%%%%%%%%%%%%%%%%%%%%%%%%%%%%%%%%%%%
\begin{frame}
\frametitle{\emph{Outwash}}
\begin{figure}
\begin{center}
\includegraphics[scale=0.48]{outwash1}
\end{center}
\end{figure}
\end{frame}
%%%%%%%%%%%%%%%%%%%%%%%%%%%%%%%%%%%%%%%%%%%%%%%%%%%%%%%%%%%%%%%%%%%%%%%%%
\begin{frame}
\frametitle{\emph{Outwash}}
\begin{figure}
\begin{center}
\includegraphics[scale=0.48]{outwash2}
\end{center}
\end{figure}
\end{frame}
%%%%%%%%%%%%%%%%%%%%%%%%%%%%%%%%%%%%%%%%%%%%%%%%%%%%%%%%%%%%%%%%%%%%%%%%%
\begin{frame}
\frametitle{Ambiente Periglacial}
\justifying
\small{
Los ambientes periglaciares se caracterizan por un predominio de los ciclos de hielo y deshielo del terreno y por la existencia de un permafrost o terreno perennemente helado que genera procesos tales como:\vfill
\textbf{Gelifracción (crioclastia)}\\
Meteorización física consecuencia del hielo que se forma en el interior de las grietas en las rocas, que producen aumento de volumen, desescamación y posterior fragmentación de la roca.\vfill
\textbf{Solifluxión}\\
Movimientos ladera abajo del suelo en regiones frías por el ciclo de hielo -deshielo.\vfill
\begin{itemize}
\item \textbf{Gelifluxión}.
Procesos de solifluxión desencadenados al saturarse la capa superior del pergelisuelo (mollisuelo) como consecuencia del deshielo por que el subsuelo o permafrost está siempre congelado y por tanto es impermeable; esto impide que el hielo fundido del mollisuelo pueda infiltrarse.\\
\item \textbf{Crioturbación}.
Forma especial de solifluxión que provoca una serie de movimientos verticales, laterales y de fisuración superficial que modifican la consistencia del suelo y favorecen su fluidificación, rompiendo su estructura y dándole un aspecto caótico.
\end{itemize}
}
\end{frame}
%%%%%%%%%%%%%%%%%%%%%%%%%%%%%%%%%%%%%%%%%%%%%%%%%%%%%%%%%%%%%%%%%%%%%%%%%%%
\begin{frame}
\frametitle{Gelifracción}
\begin{figure}
\begin{center}
\includegraphics[scale=0.48]{gelifraccion}
\end{center}
\end{figure}
\end{frame}
%%%%%%%%%%%%%%%%%%%%%%%%%%%%%%%%%%%%%%%%%%%%%%%%%%%%%%%%%%%%%%%%%%%%%%%%%
\begin{frame}
\frametitle{Gelifracción}
\begin{figure}
\begin{center}
\includegraphics[scale=0.48]{gelifraccion1}
\end{center}
\end{figure}
\tiny{\url{Fuente: http://www.redes-cepalcala.org/ciencias1/geologia/islandia/geologia.islandia_periglaciarismo.htm}}
\end{frame}
%%%%%%%%%%%%%%%%%%%%%%%%%%%%%%%%%%%%%%%%%%%%%%%%%%%%%%%%%%%%%%%%%%%%%%%%%
\begin{frame}
\frametitle{Gelifluxión}
\begin{figure}
\begin{center}
\includegraphics[scale=0.5]{gelifluxion}
\end{center}
\end{figure}
\end{frame}
%%%%%%%%%%%%%%%%%%%%%%%%%%%%%%%%%%%%%%%%%%%%%%%%%%%%%%%%%%%%%%%%%%%%%%%%%
\begin{frame}
\frametitle{Gelifluxión}
\begin{figure}
\begin{center}
\includegraphics[scale=0.48]{gelifluxion1}
\end{center}
\end{figure}
\end{frame}
%%%%%%%%%%%%%%%%%%%%%%%%%%%%%%%%%%%%%%%%%%%%%%%%%%%%%%%%%%%%%%%%%%%%%%%%%
\begin{frame}
\frametitle{Gelifluxión}
\begin{figure}
\begin{center}
\includegraphics[scale=0.51]{gelifluxion2}
\end{center}
\end{figure}
\tiny{\url{Fuente: http://www.redes-cepalcala.org/ciencias1/geologia/islandia/geologia.islandia_periglaciarismo.htm}}
\end{frame}
%%%%%%%%%%%%%%%%%%%%%%%%%%%%%%%%%%%%%%%%%%%%%%%%%%%%%%%%%%%%%%%%%%%%%%%%%
\begin{frame}
\frametitle{Crioturbación}
\begin{figure}
\begin{center}
\includegraphics[scale=0.48]{crioturbacion}
\end{center}
\end{figure}
\tiny{\url{Fuente: http://www.redes-cepalcala.org/ciencias1/geologia/islandia/geologia.islandia_periglaciarismo.htm}}
\end{frame}
%%%%%%%%%%%%%%%%%%%%%%%%%%%%%%%%%%%%%%%%%%%%%%%%%%%%%%%%%%%%%%%%%%%%%%%%%

\end{document}