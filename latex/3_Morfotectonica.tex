\documentclass{beamer}

\usepackage{beamerthemeCambridgeUS}
\usepackage{textpos}
\usepackage{ragged2e}

\graphicspath{{G:/My Drive/FIGURAS/}}

\title[Morfotectónica]{GEOMORFOLOGÍA}
\author[Edier Aristizabal]{Edier V. Aristizábal G.}
\institute{evaristizabalg@unal.edu.co}
\date{Versión:\today}

\addtobeamertemplate{headline}{}{%
	\begin{textblock*}{2mm}(.9\textwidth,0cm)
	\hfill\includegraphics[height=1cm]{un}  
	\end{textblock*}}

\begin{document}
%%%%%%%%%%%%%%%%%%%%%%%%%%%%%%%%%%%%%%%%%%%%%%%%%%%%%%%%%%%%%%%%%%%%%%%
\begin{frame}
\titlepage
\centering
	\includegraphics[width=5cm]{unal}\hspace*{4.75cm}~%
   	\includegraphics[width=2cm]{logo3} 
\end{frame}
%%%%%%%%%%%%%%%%%%%%%%%%%%%%%%%%%%%%%%%%%%%%%%%%%%%%%%%%%%%%%%%%%%%%%%%
\begin{frame}
\frametitle{GEOMORFOLOGÍA TECTÓNICA \& ESTRUCTURAL}
\justifying
\small{
\textbf{Geoformas tectónicas}: son producidas por procesos endógenos sin la intervención de las fuerzas denudacionales (proceso exógenos).\vfill
\textbf{Prediseño tectónico}. Características del paisaje con características endogénicas o tectónicas estampadas sobre ellas (redes de drenaje).\vfill
La influencia tectónica se manifiesta en la estructura de las cadenas montañosas, volcanes, arcos de isla, y otras estructuras de gran escala expuesta sobre la superficie terrestre, pero también en pequeños elementos tales como escarpes de falla.\vfill
\textbf{Geomorfología tectónica}: investiga los efectos de los procesos tectónicos activos (fallas, lineamientos, subsidencias) sobre las geoformas.\vfill
\textbf{Geoformas estructurales}: resultado de las fuerza exógenas actuando sobre geoformas tectónicas denudando rocas menos resistentes o líneas de debilidad.\vfill
\textbf{Geomorfología estructural}: influencia pasiva de estructuras geológicas sobre geoformas.}\\
\begin{center}
\includegraphics[scale=0.4]{estructural}
\end{center}
\end{frame}
%%%%%%%%%%%%%%%%%%%%%%%%%%%%%%%%%%%%%%%%%%%%%%%%%%%%%%%%%%%%%%%%%%%%%%%
\begin{frame}
\frametitle{Estructura de la Tierra}
\begin{center}
\includegraphics[scale=0.40]{interiortierra}
\end{center}
\tiny{Fuente: Fundamentals of geomorphology by R. Huggett}
\end{frame}
%%%%%%%%%%%%%%%%%%%%%%%%%%%%%%%%%%%%%%%%%%%%%%%%%%%%%%%%%%%%%%%%%%%%%%
\begin{frame}
\frametitle{Estructura de la Tierra}
\begin{center}
\includegraphics[scale=0.45]{interiortierra2}
\end{center}
\tiny{}
\end{frame}
%%%%%%%%%%%%%%%%%%%%%%%%%%%%%%%%%%%%%%%%%%%%%%%%%%%%%%%%%%%%%%%%%%%%%%
\begin{frame}
\frametitle{Tectónica de Placas}
\begin{center}
\includegraphics[scale=0.5]{tectonicplates}
\end{center}
\tiny{Fuente: Fundamentals of geomorphology by R. Huggett}
\end{frame}
%%%%%%%%%%%%%%%%%%%%%%%%%%%%%%%%%%%%%%%%%%%%%%%%%%%%%%%%%%%%%%%%%%%%%%
\begin{frame}
\frametitle{Procesos Endogénicos}
\begin{center}
\includegraphics[scale=0.5]{endogenico}
\end{center}
\tiny{http://www.clearias.com/endogenic-forces/}
\end{frame}
%%%%%%%%%%%%%%%%%%%%%%%%%%%%%%%%%%%%%%%%%%%%%%%%%%%%%%%%%%%%%%%%%%%%%%
\begin{frame}
\frametitle{Isostacia}
\begin{center}
\justifying
\small{Levantamiento vertical o subsidencia de la corteza en respuesta a cambios en el espesor. A medida que material es adicionado, el espesor de la corteza aumenta y se hunde mas dentro del manto, y a medida que el material es erosionado, la corteza se adelgaza, y material formado a profundidades sube hacia la superficie}
\begin{columns}
		\begin{column}{.5\linewidth}
		 \includegraphics[width=6cm]{isos1}
		\end{column}
		\begin{column}{.5\linewidth}
			 \includegraphics[width=6cm]{isos2}
		\end{column}
\end{columns}
\end{center}
\tiny{http://www.clearias.com/endogenic-forces/}
\end{frame}
%%%%%%%%%%%%%%%%%%%%%%%%%%%%%%%%%%%%%%%%%%%%%%%%%%%%%%%%%%%%%%%%%%%%%%
\begin{frame}
\frametitle{Isostacia}
\begin{center}
\begin{columns}
		\begin{column}{.3\linewidth}
		 \includegraphics[width=3cm]{isos3}
		\end{column}
		\begin{column}{.7\linewidth}
			 \includegraphics[width=8cm]{isos4}
		\end{column}
\end{columns}
\end{center}
\end{frame}
%%%%%%%%%%%%%%%%%%%%%%%%%%%%%%%%%%%%%%%%%%%%%%%%%%%%%%%%%%%%%%%%%%%%%%
\begin{frame}
\frametitle{Isostacia}
\begin{center}
\includegraphics[scale=0.48]{isos5}
\end{center}
\end{frame}
%%%%%%%%%%%%%%%%%%%%%%%%%%%%%%%%%%%%%%%%%%%%%%%%%%%%%%%%%%%%%%%%%%%%%%
\begin{frame}
\frametitle{Isostacia}
\begin{center}
\includegraphics[scale=0.48]{isos6}
\end{center}
\end{frame}
%%%%%%%%%%%%%%%%%%%%%%%%%%%%%%%%%%%%%%%%%%%%%%%%%%%%%%%%%%%%%%%%%%%%%%
\begin{frame}
\frametitle{Levantamiento Vertical}
\begin{center}
\includegraphics[scale=0.48]{levvertical}
\end{center}
\tiny{Fuente: Tectonic geomorphology of mountains}
\end{frame}
%%%%%%%%%%%%%%%%%%%%%%%%%%%%%%%%%%%%%%%%%%%%%%%%%%%%%%%%%%%%%%%%%%%%%%
\begin{frame}
\frametitle{Geoformas al interior de la placa}
\begin{center}
\includegraphics[scale=0.40]{interior}
\end{center}
\justifying
\small{Los \textbf{cratones} son antiguas superficies continentales de baja elevación y bajo relieve que se caracterizan por su prolongada estabilidad tectónica. Cuando los cratones son expuestos se utiliza el término \textbf{escudos}, y cuando son cubiertos inconformemente por delgadas cubiertas sedimentarias se utiliza el termino \textbf{plataforma}.}
\end{frame}
%%%%%%%%%%%%%%%%%%%%%%%%%%%%%%%%%%%%%%%%%%%%%%%%%%%%%%%%%%%%%%%%%%%%%%
\begin{frame}
\frametitle{Geoformas al interior de la placa}
\begin{center}
\includegraphics[width=8cm]{plateaux}\vfill
\includegraphics[width=8cm]{plateaux2}
\end{center}
\end{frame}
%%%%%%%%%%%%%%%%%%%%%%%%%%%%%%%%%%%%%%%%%%%%%%%%%%%%%%%%%%%%%%%%%%%%%%
\begin{frame}
\frametitle{Geoformas al interior de la placa}
\begin{center}
\includegraphics[width=10cm]{plateaux3}
\end{center}
\end{frame}
%%%%%%%%%%%%%%%%%%%%%%%%%%%%%%%%%%%%%%%%%%%%%%%%%%%%%%%%%%%%%%%%%%%%%%
\begin{frame}
\frametitle{Geoformas al interior de la placa}
\begin{center}
\begin{figure}
\includegraphics[width=10cm]{plateaux4}
\caption{Columbia River Plateau in North America}
\end{figure}
\end{center}
\end{frame}
%%%%%%%%%%%%%%%%%%%%%%%%%%%%%%%%%%%%%%%%%%%%%%%%%%%%%%%%%%%%%%%%%%%%%%
\begin{frame}
\frametitle{Geoformas al interior de la placa}
\begin{center}
\begin{figure}
\includegraphics[width=10cm]{plateaux5}
\end{figure}
\end{center}
\end{frame}
%%%%%%%%%%%%%%%%%%%%%%%%%%%%%%%%%%%%%%%%%%%%%%%%%%%%%%%%%%%%%%%%%%%%%%
\begin{frame}
\frametitle{Geoformas al interior de la placa}
\begin{center}
\begin{figure}
\includegraphics[width=9cm]{plateaux6}
\caption{Merrick Butte in Monument Valley, Arizona}
\end{figure}
\end{center}
\end{frame}
%%%%%%%%%%%%%%%%%%%%%%%%%%%%%%%%%%%%%%%%%%%%%%%%%%%%%%%%%%%%%%%%%%%%%%
\begin{frame}
\frametitle{Márgenes Activas \& Pasivas}
\begin{center}
\begin{figure}
\includegraphics[width=12cm]{maractiva}
\end{figure}
\end{center}
\end{frame}
%%%%%%%%%%%%%%%%%%%%%%%%%%%%%%%%%%%%%%%%%%%%%%%%%%%%%%%%%%%%%%%%%%%%%%
\begin{frame}
\frametitle{Márgenes Pasivas}
\begin{center}
\begin{figure}
\includegraphics[width=9cm]{pasivas}
\end{figure}
\end{center}
\end{frame}
%%%%%%%%%%%%%%%%%%%%%%%%%%%%%%%%%%%%%%%%%%%%%%%%%%%%%%%%%%%%%%%%%%%%%%
\begin{frame}
\frametitle{Márgenes Pasivas}
\begin{center}
\begin{figure}
\includegraphics[width=11cm]{pasivas1}
\end{figure}
\end{center}
\end{frame}
%%%%%%%%%%%%%%%%%%%%%%%%%%%%%%%%%%%%%%%%%%%%%%%%%%%%%%%%%%%%%%%%%%%%%%
\begin{frame}
\frametitle{Márgenes Pasivas}
\begin{center}
\begin{figure}
\includegraphics[width=9cm]{pasivas2}
\caption{Escarpe y plateaux de una margen pasiva en el norte de New South Wales.}
\end{figure}
\end{center}
\end{frame}
%%%%%%%%%%%%%%%%%%%%%%%%%%%%%%%%%%%%%%%%%%%%%%%%%%%%%%%%%%%%%%%%%%%%%%
\begin{frame}
\frametitle{Márgenes Pasivas}
\begin{center}
\begin{figure}
\includegraphics[width=8cm]{pasivas3}
\end{figure}
\end{center}
\tiny{Fuente: Bishop (2007)}
\end{frame}
%%%%%%%%%%%%%%%%%%%%%%%%%%%%%%%%%%%%%%%%%%%%%%%%%%%%%%%%%%%%%%%%%%%%%%
\begin{frame}
\frametitle{Márgenes Activas}
\begin{center}
\begin{figure}
\includegraphics[width=10cm]{activas}
\end{figure}
\end{center}
\end{frame}
%%%%%%%%%%%%%%%%%%%%%%%%%%%%%%%%%%%%%%%%%%%%%%%%%%%%%%%%%%%%%%%%%%%%%%
\begin{frame}
\frametitle{Márgenes Activas}
\begin{center}
\begin{figure}
\includegraphics[width=10cm]{activas1}
\end{figure}
\end{center}
\end{frame}
%%%%%%%%%%%%%%%%%%%%%%%%%%%%%%%%%%%%%%%%%%%%%%%%%%%%%%%%%%%%%%%%%%%%%%
\begin{frame}
\frametitle{Márgenes Divergentes}
\begin{center}
\begin{figure}
\includegraphics[width=8cm]{divergentes}
\end{figure}
\end{center}
\tiny{http://www.csus.edu/indiv/k/kusnickj/geology12/tectonicbasins.html}
\end{frame}
%%%%%%%%%%%%%%%%%%%%%%%%%%%%%%%%%%%%%%%%%%%%%%%%%%%%%%%%%%%%%%%%%%%%%%
\begin{frame}
\frametitle{Rift Continental}
\framesubtitle{fase I $\rightarrow$ cadena de valles y montañas lineales}
\begin{center}
\begin{columns}
		\begin{column}{.5\linewidth}
		 \includegraphics[width=6cm]{fase1}
		\end{column}
		\begin{column}{.5\linewidth}
			 \includegraphics[width=6cm]{fase1b}
		\end{column}
\end{columns}
\end{center}
\end{frame}
%%%%%%%%%%%%%%%%%%%%%%%%%%%%%%%%%%%%%%%%%%%%%%%%%%%%%%%%%%%%%%%%%%%%%%
\begin{frame}
\frametitle{Rift Continental}
\framesubtitle{fase II $\rightarrow$ cuenca continental}
\begin{center}
\begin{columns}
		\begin{column}{.4\linewidth}
		 \includegraphics[width=6cm]{fase2}
		\end{column}
		\begin{column}{.5\linewidth}
			 \includegraphics[width=6cm]{fase2b}
		\end{column}
\end{columns}
\end{center}
\end{frame}
%%%%%%%%%%%%%%%%%%%%%%%%%%%%%%%%%%%%%%%%%%%%%%%%%%%%%%%%%%%%%%%%%%%%%%
\begin{frame}
\frametitle{Rift Continental}
\framesubtitle{fase III $\rightarrow$ rift oceánico}
\begin{center}
\begin{columns}
		\begin{column}{.4\linewidth}
		 \includegraphics[width=6cm]{fase4}
		\end{column}
		\begin{column}{.5\linewidth}
			 \includegraphics[width=4cm]{fase4b}
		\end{column}
\end{columns}
\end{center}
\end{frame}
%%%%%%%%%%%%%%%%%%%%%%%%%%%%%%%%%%%%%%%%%%%%%%%%%%%%%%%%%%%%%%%%%%%%%%
\begin{frame}
\frametitle{Rift Continental}
\framesubtitle{fase IV $\rightarrow$ rift oceánico}
\begin{center}
\begin{columns}
		\begin{column}{.7\linewidth}
		 \includegraphics[width=7cm]{fase3}
		\end{column}
		\begin{column}{.3\linewidth}
			 \includegraphics[width=4cm]{fase3b}
		\end{column}
\end{columns}
\end{center}
\end{frame}
%%%%%%%%%%%%%%%%%%%%%%%%%%%%%%%%%%%%%%%%%%%%%%%%%%%%%%%%%%%%%%%%%%%%%%
\begin{frame}
\frametitle{Márgenes Convergentes}
\begin{center}
\begin{figure}
\includegraphics[width=10cm]{convergente}
\end{figure}
\end{center}
\end{frame}
%%%%%%%%%%%%%%%%%%%%%%%%%%%%%%%%%%%%%%%%%%%%%%%%%%%%%%%%%%%%%%%%%%%%%%
\begin{frame}
\frametitle{Oceano / Oceano}
\framesubtitle{Ejemplos: Japón, Aleutians}
\begin{center}
\begin{figure}
\includegraphics[width=10cm]{convergente1}
\end{figure}
\end{center}
\end{frame}
%%%%%%%%%%%%%%%%%%%%%%%%%%%%%%%%%%%%%%%%%%%%%%%%%%%%%%%%%%%%%%%%%%%%%%
\begin{frame}
\frametitle{Oceano / Oceano}
\framesubtitle{Arco Volcánico Insular}
\begin{center}
\begin{figure}
\includegraphics[width=8cm]{insular}
\end{figure}
\end{center}
\end{frame}
%%%%%%%%%%%%%%%%%%%%%%%%%%%%%%%%%%%%%%%%%%%%%%%%%%%%%%%%%%%%%%%%%%%%%%
\begin{frame}
\frametitle{Oceano / Oceano}
\begin{center}
\begin{columns}
		\begin{column}{.5\linewidth}
		 \includegraphics[width=6cm]{oceano1}
		\end{column}
		\begin{column}{.5\linewidth}
			 \includegraphics[width=4.5cm]{oceano2}\vfill
			 \includegraphics[width=4.5cm]{oceano3}
		\end{column}
\end{columns}
\end{center}
\end{frame}
%%%%%%%%%%%%%%%%%%%%%%%%%%%%%%%%%%%%%%%%%%%%%%%%%%%%%%%%%%%%%%%%%%%%%%
\begin{frame}
\frametitle{Oceano / Continente}
\framesubtitle{Andes, Cascadearcocontinental}
\begin{center}
\begin{figure}
\includegraphics[width=10cm]{ocecont}
\end{figure}
\end{center}
\end{frame}
%%%%%%%%%%%%%%%%%%%%%%%%%%%%%%%%%%%%%%%%%%%%%%%%%%%%%%%%%%%%%%%%%%%%%%
\begin{frame}
\frametitle{Arco Volcánico Continental}
\begin{center}
\begin{figure}
\includegraphics[width=9cm]{arcocontinental}
\end{figure}
\end{center}
\end{frame}
%%%%%%%%%%%%%%%%%%%%%%%%%%%%%%%%%%%%%%%%%%%%%%%%%%%%%%%%%%%%%%%%%%%%%%
\begin{frame}
\frametitle{Arco Volcánico Continental}
\framesubtitle{Cascade}
\begin{center}
\begin{columns}
		\begin{column}{.6\linewidth}
		 \includegraphics[width=6cm]{arcocontinental1}
		\end{column}
		\begin{column}{.4\linewidth}
			 \includegraphics[width=4cm]{oregon}
		\end{column}
\end{columns}
\end{center}
\end{frame}
%%%%%%%%%%%%%%%%%%%%%%%%%%%%%%%%%%%%%%%%%%%%%%%%%%%%%%%%%%%%%%%%%%%%%%
\begin{frame}
\frametitle{Arco Volcánico Continental}
\framesubtitle{Cascade}
\begin{center}
\begin{figure}
\includegraphics[width=11cm]{olympic}
\caption{Olympic National Park (NW USA)tcont.}
\end{figure}
\end{center}
\end{frame}
%%%%%%%%%%%%%%%%%%%%%%%%%%%%%%%%%%%%%%%%%%%%%%%%%%%%%%%%%%%%%%%%%%%%%%
\begin{frame}
\frametitle{Continente / Continente}
\framesubtitle{Ejemplos: Himalaya, Alpes, Apalaches}
\begin{center}
\begin{figure}
\includegraphics[width=10cm]{contcont}
\caption{Olympic National Park (NW USA)tcont.}
\end{figure}
\end{center}
\end{frame}
%%%%%%%%%%%%%%%%%%%%%%%%%%%%%%%%%%%%%%%%%%%%%%%%%%%%%%%%%%%%%%%%%%%%%%
\begin{frame}
\frametitle{Continente / Continente}
\framesubtitle{Himalaya}
\begin{center}
\begin{columns}
		\begin{column}{.4\linewidth}
		 \includegraphics[width=4cm]{himalayas}
		\end{column}
		\begin{column}{.6\linewidth}
			 \includegraphics[width=7cm]{himalayas1}
		\end{column}
\end{columns}
\end{center}
\end{frame}
%%%%%%%%%%%%%%%%%%%%%%%%%%%%%%%%%%%%%%%%%%%%%%%%%%%%%%%%%%%%%%%%%%%%%%
\begin{frame}
\frametitle{Márgenes Transformantes}
\begin{center}
\begin{figure}
\includegraphics[width=9cm]{transformantes}
\end{figure}
\end{center}
\end{frame}
%%%%%%%%%%%%%%%%%%%%%%%%%%%%%%%%%%%%%%%%%%%%%%%%%%%%%%%%%%%%%%%%%%%%%%
\begin{frame}
\frametitle{Márgenes Transformantes}
\framesubtitle{Falla San Andrésotspots}
\begin{center}
\begin{figure}
\includegraphics[width=9cm]{sanandres}
\end{figure}
\end{center}
\end{frame}
%%%%%%%%%%%%%%%%%%%%%%%%%%%%%%%%%%%%%%%%%%%%%%%%%%%%%%%%%%%%%%%%%%%%%%
\begin{frame}
\frametitle{Plumas del Manto (\emph{hotspots})}
\begin{center}
\begin{figure}
\includegraphics[width=10cm]{hotspots}
\end{figure}
\end{center}
\end{frame}
%%%%%%%%%%%%%%%%%%%%%%%%%%%%%%%%%%%%%%%%%%%%%%%%%%%%%%%%%%%%%%%%%%%%%%
\begin{frame}
\frametitle{Continental \emph{flood basalts}}
\begin{center}
\begin{figure}
\includegraphics[width=10cm]{floodbasalts}
\end{figure}
\end{center}
\end{frame}
%%%%%%%%%%%%%%%%%%%%%%%%%%%%%%%%%%%%%%%%%%%%%%%%%%%%%%%%%%%%%%%%%%%%%%
\begin{frame}
\begin{center}
\begin{figure}
\includegraphics[width=10cm]{floodbasalts1}
\end{figure}
\end{center}
\end{frame}
%%%%%%%%%%%%%%%%%%%%%%%%%%%%%%%%%%%%%%%%%%%%%%%%%%%%%%%%%%%%%%%%%%%%%%
\begin{frame}
\frametitle{Cadena Islas Volcánicas}
\begin{center}
\begin{figure}
\includegraphics[width=7cm]{cadena}
\end{figure}
\end{center}
\end{frame}
%%%%%%%%%%%%%%%%%%%%%%%%%%%%%%%%%%%%%%%%%%%%%%%%%%%%%%%%%%%%%%%%%%%%%%
\begin{frame}
\begin{center}
\begin{figure}
\includegraphics[width=10cm]{cadenas1}
\end{figure}
\end{center}
\end{frame}
%%%%%%%%%%%%%%%%%%%%%%%%%%%%%%%%%%%%%%%%%%%%%%%%%%%%%%%%%%%%%%%%%%%%%%
\begin{frame}
\begin{center}
\begin{figure}
\includegraphics[width=12cm]{morfotectonica}
\end{figure}
\end{center}
\end{frame}
%%%%%%%%%%%%%%%%%%%%%%%%%%%%%%%%%%%%%%%%%%%%%%%%%%%%%%%%%%%%%%%%%%%%%%
\end{document}
