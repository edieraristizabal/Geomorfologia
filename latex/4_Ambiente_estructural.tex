\documentclass{beamer}

\usepackage{beamerthemeCambridgeUS}
\usepackage{textpos}
\usepackage{ragged2e}

\graphicspath{{G:/My Drive/FIGURAS/}}

\title[Ambiente Estructural]{GEOMORFOLOGÍA}
\author[Edier Aristizabal]{Edier V. Aristizábal G.}
\institute{evaristizabalg@unal.edu.co}
\date{Versión:\today}

\addtobeamertemplate{headline}{}{%
	\begin{textblock*}{2mm}(.9\textwidth,0cm)
	\hfill\includegraphics[height=1cm]{un}  
	\end{textblock*}}

\begin{document}
%%%%%%%%%%%%%%%%%%%%%%%%%%%%%%%%%%%%%%%%%%%%%%%%%%%%%%%%%%%%%%%%%%%%%%%%%
\begin{frame}
\titlepage
\centering
	\includegraphics[width=5cm]{unal}\hspace*{4.75cm}~%
   	\includegraphics[width=2cm]{logo3} 
\end{frame}
%%%%%%%%%%%%%%%%%%%%%%%%%%%%%%%%%%%%%%%%%%%%%%%%%%%%%%%%%%%%%%%%%%%%%%%%%
\begin{frame}
\frametitle{GEOMORFOLOGÍA TECTÓNICA \& ESTRUCTURAL}
\justifying
\small{
\textbf{Geoformas tectónicas}: son producidas por procesos endógenos sin la intervención de las fuerzas denudacionales (proceso exógenos).\vfill
\textbf{Prediseño tectónico}. Características del paisaje con características endogénicas o tectónicas estampadas sobre ellas (redes de drenaje).\vfill
La influencia tectónica se manifiesta en la estructura de las cadenas montañosas, volcanes, arcos de isla, y otras estructuras de gran escala expuesta sobre la superficie terrestre, pero también en pequeños elementos tales como escarpes de falla.\vfill
\textbf{Geomorfología tectónica}: investiga los efectos de los procesos tectónicos activos (fallas, lineamientos, subsidencias) sobre las geoformas.\vfill
\textbf{Geoformas estructurales}: resultado de las fuerza exógenas actuando sobre geoformas tectónicas denudando rocas menos resistentes o líneas de debilidad.\vfill
\textbf{Geomorfología estructural}: influencia pasiva de estructuras geológicas sobre geoformas.}\\
\begin{center}
\includegraphics[scale=0.4]{estructural}
\end{center}
\end{frame}
%%%%%%%%%%%%%%%%%%%%%%%%%%%%%%%%%%%%%%%%%%%%%%%%%%%%%%%%%%%%%%%%%%%%%%%
\begin{frame}
\begin{center}
\includegraphics[scale=0.55]{esfuerzos}
\end{center}
\end{frame}
%%%%%%%%%%%%%%%%%%%%%%%%%%%%%%%%%%%%%%%%%%%%%%%%%%%%%%%%%%%%%%%%%%%%%%%
\begin{frame}
\frametitle{Estructuras Geológicas}
\begin{center}
\includegraphics[scale=0.55]{estructurasgeologicas}
\end{center}
\end{frame}
%%%%%%%%%%%%%%%%%%%%%%%%%%%%%%%%%%%%%%%%%%%%%%%%%%%%%%%%%%%%%%%%%%%%%%%
\begin{frame}
\frametitle{Rumbo y Buzamiento}
\centering
\begin{columns}
	\begin{column}{.5\linewidth}
	\includegraphics[scale=0.5]{rumbo}
	\end{column}
	\begin{column}{.5\linewidth}
	\includegraphics[scale=0.6]{strike}\vfill
	\includegraphics[scale=0.6]{plano}
	\end{column}
\end{columns}
\end{frame}
%%%%%%%%%%%%%%%%%%%%%%%%%%%%%%%%%%%%%%%%%%%%%%%%%%%%%%%%%%%%%%%%%%%%%%%
\begin{frame}
\frametitle{Relieve fallado}
\centering
\begin{columns}
	\begin{column}{.5\linewidth}
	\includegraphics[scale=0.5]{normal}\vfill
	\includegraphics[scale=0.5]{thrust}
	\end{column}
	\begin{column}{.5\linewidth}
	\includegraphics[scale=0.5]{reverse}\vfill
	\includegraphics[scale=0.5]{strikeslip}
	\end{column}
\end{columns}
\tiny{Harry Williams, Geomorphology}
\end{frame}
%%%%%%%%%%%%%%%%%%%%%%%%%%%%%%%%%%%%%%%%%%%%%%%%%%%%%%%%%%%%%%%%%%%%%%%
\begin{frame}
\frametitle{Relieve fallado}
\begin{center}
\includegraphics[scale=0.55]{fault}
\end{center}
\end{frame}
%%%%%%%%%%%%%%%%%%%%%%%%%%%%%%%%%%%%%%%%%%%%%%%%%%%%%%%%%%%%%%%%%%%%%%%
\begin{frame}
\frametitle{Geoformas Relieve Fallado}
\framesubtitle{Marco extensional}
\begin{center}
\includegraphics[scale=0.52]{extension}
\end{center}
\end{frame}
%%%%%%%%%%%%%%%%%%%%%%%%%%%%%%%%%%%%%%%%%%%%%%%%%%%%%%%%%%%%%%%%%%%%%%%
\begin{frame}
\frametitle{Geoformas Relieve Fallado}
\framesubtitle{Marco extensional}
\begin{center}
\includegraphics[scale=0.40]{graven}\vfill
\includegraphics[scale=0.52]{gravenhorst}
\end{center}
\end{frame}
%%%%%%%%%%%%%%%%%%%%%%%%%%%%%%%%%%%%%%%%%%%%%%%%%%%%%%%%%%%%%%%%%%%%%%%
\begin{frame}
\frametitle{Geoformas Relieve Fallado}
\framesubtitle{Marco extensional}
\begin{center}
\includegraphics[scale=0.55]{grabenhorst1}
\end{center}
\end{frame}
%%%%%%%%%%%%%%%%%%%%%%%%%%%%%%%%%%%%%%%%%%%%%%%%%%%%%%%%%%%%%%%%%%%%%%%
\begin{frame}
\frametitle{Geoformas Relieve Fallado}
\framesubtitle{Marco extensional}
\begin{center}
\includegraphics[scale=0.50]{grabenhorst2}
\end{center}
\end{frame}
%%%%%%%%%%%%%%%%%%%%%%%%%%%%%%%%%%%%%%%%%%%%%%%%%%%%%%%%%%%%%%%%%%%%%%%
\begin{frame}
\frametitle{Geoformas Relieve Fallado}
\framesubtitle{Marco extensional}
\begin{center}
\includegraphics[scale=0.50]{grabenhorst3}
\end{center}
\end{frame}
%%%%%%%%%%%%%%%%%%%%%%%%%%%%%%%%%%%%%%%%%%%%%%%%%%%%%%%%%%%%%%%%%%%%%%%
\begin{frame}
\frametitle{Geoformas Relieve Fallado}
\framesubtitle{Marco extensional}
\centering
\includegraphics[scale=0.4]{valle}
\end{frame}
%%%%%%%%%%%%%%%%%%%%%%%%%%%%%%%%%%%%%%%%%%%%%%%%%%%%%%%%%%%%%%%%%%%%%%%
\begin{frame}
\frametitle{Geoformas fallas de componente vertical}
\framesubtitle{Expresión topográfica: facetas triangulares}
\centering
\includegraphics[scale=0.55]{facetas}
\end{frame}
%%%%%%%%%%%%%%%%%%%%%%%%%%%%%%%%%%%%%%%%%%%%%%%%%%%%%%%%%%%%%%%%%%%%%%%
\begin{frame}
\frametitle{Geoformas fallas de componente vertical}
\framesubtitle{Expresión topográfica: facetas triangulares}
\centering
\includegraphics[scale=0.50]{sanandres2}
\end{frame}
%%%%%%%%%%%%%%%%%%%%%%%%%%%%%%%%%%%%%%%%%%%%%%%%%%%%%%%%%%%%%%%%%%%%%%%
\begin{frame}
\frametitle{Geoformas fallas de componente vertical}
\framesubtitle{Expresión topográfica: facetas triangulares}
\centering
\includegraphics[scale=0.50]{facetas2}
\end{frame}
%%%%%%%%%%%%%%%%%%%%%%%%%%%%%%%%%%%%%%%%%%%%%%%%%%%%%%%%%%%%%%%%%%%%%%%
\begin{frame}
\frametitle{Geoformas fallas de componente vertical}
\framesubtitle{Expresión topográfica: facetas triangulares}
\centering
\includegraphics[scale=0.40]{facetas3}
\end{frame}
%%%%%%%%%%%%%%%%%%%%%%%%%%%%%%%%%%%%%%%%%%%%%%%%%%%%%%%%%%%%%%%%%%%%%%%
\begin{frame}
\frametitle{Geoformas fallas de componente vertical}
\framesubtitle{Expresión topográfica: facetas triangulares}
\centering
\includegraphics[scale=0.45]{facetas4}
\end{frame}
%%%%%%%%%%%%%%%%%%%%%%%%%%%%%%%%%%%%%%%%%%%%%%%%%%%%%%%%%%%%%%%%%%%%%%%
\begin{frame}
\frametitle{Geoformas fallas de componente vertical}
\framesubtitle{Inversión relieve}
\centering
\includegraphics[scale=0.6]{inversionrelieve}
\end{frame}
%%%%%%%%%%%%%%%%%%%%%%%%%%%%%%%%%%%%%%%%%%%%%%%%%%%%%%%%%%%%%%%%%%%%%%%
\begin{frame}
\frametitle{Geoformas fallas de componente rumbo}
\framesubtitle{Desplazamiento del relieve}
\centering
\includegraphics[scale=0.65]{desplazamientorelieve}
\end{frame}
%%%%%%%%%%%%%%%%%%%%%%%%%%%%%%%%%%%%%%%%%%%%%%%%%%%%%%%%%%%%%%%%%%%%%%%
\begin{frame}
\frametitle{Geoformas fallas de componente rumbo}
\framesubtitle{Desplazamiento del relieve}
\justifying
\textbf{Lomos de obturación} (\emph{shutter ridges}) y drenajes desplazados (\emph{offset drainage})\vfill
\textbf{Lomos de presión} (\emph{pressure ridges}) y lagos de falla (\emph{sag pond})
\centering
\includegraphics[scale=0.75]{desplazamiento}
\end{frame}
%%%%%%%%%%%%%%%%%%%%%%%%%%%%%%%%%%%%%%%%%%%%%%%%%%%%%%%%%%%%%%%%%%%%%%%
\begin{frame}
\frametitle{Geoformas fallas de componente rumbo}
\framesubtitle{Desplazamiento del relieve}
\centering
\includegraphics[scale=0.65]{linearridge}
\end{frame}
%%%%%%%%%%%%%%%%%%%%%%%%%%%%%%%%%%%%%%%%%%%%%%%%%%%%%%%%%%%%%%%%%%%%%%%
\begin{frame}
\frametitle{Geoformas fallas de componente rumbo}
\framesubtitle{Lomos de presión vs \emph{sag ponds}}
\centering
\includegraphics[scale=0.5]{pressureridge}
\begin{columns}
	\begin{column}{.5\linewidth}
	\includegraphics[scale=0.5]{sanjacinto}
	\end{column}
	\begin{column}{.5\linewidth}
	\includegraphics[scale=0.5]{sagponds}
	\end{column}
\end{columns}
\end{frame}
%%%%%%%%%%%%%%%%%%%%%%%%%%%%%%%%%%%%%%%%%%%%%%%%%%%%%%%%%%%%%%%%%%%%%%%
\begin{frame}
\frametitle{Geoformas fallas de componente rumbo}
\framesubtitle{Valles alineados}
\centering
\begin{figure}
\includegraphics[scale=0.4]{greatglen}
\caption{The Great Glen (Scotland)}
\end{figure}
\end{frame}
%%%%%%%%%%%%%%%%%%%%%%%%%%%%%%%%%%%%%%%%%%%%%%%%%%%%%%%%%%%%%%%%%%%%%%%
\begin{frame}
\frametitle{Geoformas fallas de componente rumbo}
\framesubtitle{Boquerones,Delgaditas,\emph{saddle}}
\begin{columns}
	\begin{column}{.5\linewidth}
	\centering
	\includegraphics[scale=0.5]{boqueron}
	\end{column}
	\begin{column}{.5\linewidth}
	\centering
	\includegraphics[scale=0.5]{saddle}
	\end{column}
\end{columns}
\centering
\includegraphics[scale=0.4]{topographicsaddle}
\end{frame}
%%%%%%%%%%%%%%%%%%%%%%%%%%%%%%%%%%%%%%%%%%%%%%%%%%%%%%%%%%%%%%%%%%%%%%%
\begin{frame}
\frametitle{Geoformas fallas de componente rumbo}
\framesubtitle{Boqueron}
\centering
\includegraphics[scale=0.50]{boqueron2}
\end{frame}
%%%%%%%%%%%%%%%%%%%%%%%%%%%%%%%%%%%%%%%%%%%%%%%%%%%%%%%%%%%%%%%%%%%%%%%
\begin{frame}
\frametitle{Comportamiento Ductil}
\framesubtitle{Relieve Plegado}
\centering
\includegraphics[scale=0.50]{plegado}
\end{frame}
%%%%%%%%%%%%%%%%%%%%%%%%%%%%%%%%%%%%%%%%%%%%%%%%%%%%%%%%%%%%%%%%%%%%%%%
\begin{frame}
\frametitle{Comportamiento Ductil}
\framesubtitle{Relieve Plegado}
\centering
\includegraphics[scale=0.50]{anticlinal}
\end{frame}
%%%%%%%%%%%%%%%%%%%%%%%%%%%%%%%%%%%%%%%%%%%%%%%%%%%%%%%%%%%%%%%%%%%%
\begin{frame}
\frametitle{Relieve Plegado}
\framesubtitle{Relieve Jurásico}
\small{El relieve es un directo reflejo de las estructuras que subyacen. Se forma en cordilleras jóvenes por una alternancia de pliegues convexos (anticlinales) y cóncavos (sinclinales). En los anticlinales la erosión del agua crea valles perpendiculares (cluses) y valles paralelos (combes)}
\centering
\includegraphics[scale=0.40]{jurasico}
\end{frame}
%%%%%%%%%%%%%%%%%%%%%%%%%%%%%%%%%%%%%%%%%%%%%%%%%%%%%%%%%%%%%%%%%%%%
\begin{frame}
\frametitle{Relieve Plegado}
\centering
\includegraphics[scale=0.55]{relievejurasico}
\end{frame}
%%%%%%%%%%%%%%%%%%%%%%%%%%%%%%%%%%%%%%%%%%%%%%%%%%%%%%%%%%%%%%%%%%%%
\begin{frame}
\frametitle{Relieve Plegado}
\begin{figure}
\begin{center}
\includegraphics[scale=0.50]{losapalaches}
\caption{Cordillera de Los Apalaches (EEUU)}
\end{center}
\end{figure}
\end{frame}
%%%%%%%%%%%%%%%%%%%%%%%%%%%%%%%%%%%%%%%%%%%%%%%%%%%%%%%%%%%%%%%%%%%%
\begin{frame}
\frametitle{Geoformas Estructurales}
\begin{center}
\includegraphics[scale=0.50]{geoformasestructurales}
\end{center}
\end{frame}
%%%%%%%%%%%%%%%%%%%%%%%%%%%%%%%%%%%%%%%%%%%%%%%%%%%%%%%%%%%%%%%%%%%%
\begin{frame}
\frametitle{Geoformas Estructurales}
\framesubtitle{Cerros testigos}
\begin{center}
\includegraphics[scale=0.50]{cerrostestigos}
\end{center}
\end{frame}
%%%%%%%%%%%%%%%%%%%%%%%%%%%%%%%%%%%%%%%%%%%%%%%%%%%%%%%%%%%%%%%%%%%%
\begin{frame}
\frametitle{Geoformas Estructurales}
\framesubtitle{Pinnacle}
\begin{figure}
\begin{center}
\includegraphics[scale=0.50]{pinnacle}
\caption{Pinnacle Peak, Scottsdale, Arizona}
\end{center}
\end{figure}
\end{frame}
%%%%%%%%%%%%%%%%%%%%%%%%%%%%%%%%%%%%%%%%%%%%%%%%%%%%%%%%%%%%%%%%%%%%
\begin{frame}
\frametitle{Geoformas Estructurales}
\framesubtitle{Butte}
\begin{figure}
\begin{center}
\includegraphics[scale=0.50]{butte}
\caption{Merrick Butte in Monument Valley, Arizona}
\end{center}
\end{figure}
\end{frame}
%%%%%%%%%%%%%%%%%%%%%%%%%%%%%%%%%%%%%%%%%%%%%%%%%%%%%%%%%%%%%%%%%%%%
\begin{frame}
\frametitle{Geoformas Estructurales}
\framesubtitle{Mesa}
\begin{figure}
\begin{center}
\includegraphics[scale=0.50]{mesa}
\caption{Colorado River in northern Utah}
\end{center}
\end{figure}
\end{frame}
%%%%%%%%%%%%%%%%%%%%%%%%%%%%%%%%%%%%%%%%%%%%%%%%%%%%%%%%%%%%%%%%%%%%
\begin{frame}
\frametitle{Geoformas Estructurales}
\framesubtitle{Plateaux}
\begin{figure}
\begin{center}
\includegraphics[scale=0.60]{plateaux}
\caption{Columbia River Plateaux in North America}
\end{center}
\end{figure}
\end{frame}
%%%%%%%%%%%%%%%%%%%%%%%%%%%%%%%%%%%%%%%%%%%%%%%%%%%%%%%%%%%%%%%%%%%%
\begin{frame}
\frametitle{Geoformas Estructurales}
\framesubtitle{Cuesta}
\begin{center}
\includegraphics[scale=0.60]{cuesta}
\end{center}
\end{frame}
%%%%%%%%%%%%%%%%%%%%%%%%%%%%%%%%%%%%%%%%%%%%%%%%%%%%%%%%%%%%%%%%%%%%
\begin{frame}
\frametitle{Geoformas Estructurales}
\framesubtitle{Hogbacks}
\begin{center}
\includegraphics[scale=0.55]{hogbacks}
\end{center}
\end{frame}
%%%%%%%%%%%%%%%%%%%%%%%%%%%%%%%%%%%%%%%%%%%%%%%%%%%%%%%%%%%%%%%%%%%%
\begin{frame}
\frametitle{Geoformas Estructurales}
\framesubtitle{Razorbacks}
\begin{figure}
\begin{center}
\includegraphics[scale=0.45]{razorbacks}
\caption{Crestas en calizas (Monterrey, México)}
\end{center}
\end{figure}
\end{frame}
%%%%%%%%%%%%%%%%%%%%%%%%%%%%%%%%%%%%%%%%%%%%%%%%%%%%%%%%%%%%%%%%%%%%
\begin{frame}
\frametitle{Relación Laderas vs Estructuras}
\begin{center}
\includegraphics[scale=0.58]{cataclinal}
\end{center}
\tiny{Fuente: Meentemeyer \& Moody (2000)}
\end{frame}
%%%%%%%%%%%%%%%%%%%%%%%%%%%%%%%%%%%%%%%%%%%%%%%%%%%%%%%%%%%%%%%%%%%%
\begin{frame}
\frametitle{Influencia de las Estructuras en los drenajes}
\small{
\textbf{Consecuente} (c) : que desarrolló el cauce sobre la superficie original siguiendo la pendiente regional.\\
\textbf{Resecuente} (r) : tributarios a drenajes subsecuentes que siguen la pendiente regional o sentido del buzamiento (REnewed conSEQUENT).\\
\textbf{Obsecuente} (o) : tributarios que fluye en dirección opuesta a la pendiente regional o el buzamiento o drenajes resecuentes (Opposite to conSEQUENT).\\
\textbf{Subsecuente} (s) : que desarrolló su curso ajustado a lo largo del rumbo línea menor resistencia.\\
\textbf{Insecuente} (i) : que sigue un curso sin control aparente.
}
\begin{center}
\includegraphics[scale=0.58]{drenajes}
\end{center}
\tiny{Fuente: Dirik (2005)}
\end{frame}
%%%%%%%%%%%%%%%%%%%%%%%%%%%%%%%%%%%%%%%%%%%%%%%%%%%%%%%%%%%%%%%%%%%%
\begin{frame}
\frametitle{Modelado Granítico - Estructural}
\begin{center}
\includegraphics[scale=0.4]{granitico}
\end{center}
\end{frame}
%%%%%%%%%%%%%%%%%%%%%%%%%%%%%%%%%%%%%%%%%%%%%%%%%%%%%%%%%%%%%%%%%%%%
\begin{frame}
\frametitle{Modelado Granítico - Estructural}
\begin{columns}
	\begin{column}{.5\linewidth}
	\centering
	\includegraphics[scale=0.45]{granitico2}
	\end{column}
	\begin{column}{.5\linewidth}
	\centering
	\includegraphics[scale=0.45]{granitico3}
	\end{column}
\end{columns}
\end{frame}
%%%%%%%%%%%%%%%%%%%%%%%%%%%%%%%%%%%%%%%%%%%%%%%%%%%%%%%%%%%%%%%%%%%%%%%
\begin{frame}
\frametitle{Modelos de formación de Inselbergs}
\begin{center}
\includegraphics[scale=0.4]{inselberges}
\end{center}
\tiny{Fuente: Twidale (1971) Gunnell et al (2007)}
\end{frame}
%%%%%%%%%%%%%%%%%%%%%%%%%%%%%%%%%%%%%%%%%%%%%%%%%%%%%%%%%%%%%%%%%%%%
\begin{frame}
\frametitle{Modelos de formación de Inselbergs}
\framesubtitle{Inselbergs, monadnock y bornhardts}
\small{
\textbf{Inselbergs}: montaña isla (alemán). Montañas o conjuntos de estas que destacan abruptamente de las llanuras que los rodean. Se caracterizan por tener laderas limitantes inclinadas que enlazan con los llanos adyacentes mediante uniones bien marcadas.\\
\textbf{Bornhardts}: Inselberg de forma dómica, Castillo koppies (formas acastilladas): pequeños y angulares, Nubbins o Knolls: bloques caóticos.
}
\begin{columns}
	\begin{column}{.5\linewidth}
	\begin{center}
	\includegraphics[scale=0.65]{bornhardts}
	\end{center}
	\end{column}
	\begin{column}{.5\linewidth}
	\begin{center}
	\includegraphics[scale=0.1]{inselberg}
	\end{center}
	\end{column}
\end{columns}
\end{frame}
%%%%%%%%%%%%%%%%%%%%%%%%%%%%%%%%%%%%%%%%%%%%%%%%%%%%%%%%%%%%%%%%%%%%
\begin{frame}
\frametitle{Modelado Granítico - Estructural}
\framesubtitle{Domo}
\begin{center}
\includegraphics[scale=0.85]{domo}
\end{center}
\end{frame}
%%%%%%%%%%%%%%%%%%%%%%%%%%%%%%%%%%%%%%%%%%%%%%%%%%%%%%%%%%%%%%%%%%%%
\begin{frame}
\frametitle{Modelado Granítico - Estructural}
\framesubtitle{Pieda caballete}
\begin{center}
\includegraphics[scale=0.85]{caballete}
\end{center}
\end{frame}
%%%%%%%%%%%%%%%%%%%%%%%%%%%%%%%%%%%%%%%%%%%%%%%%%%%%%%%%%%%%%%%%%%%%
\begin{frame}
\frametitle{Modelado Granítico - Estructural}
\framesubtitle{Tor}
\begin{center}
\includegraphics[scale=0.85]{tor}
\end{center}
\end{frame}
%%%%%%%%%%%%%%%%%%%%%%%%%%%%%%%%%%%%%%%%%%%%%%%%%%%%%%%%%%%%%%%%%%%%
\begin{frame}
\frametitle{Modelado Granítico - Estructural}
\framesubtitle{Nubbins, Knolls, Compayrés}
\begin{center}
\begin{figure}
\includegraphics[scale=0.45]{nubbins}
\caption{Mt Lofty Ranges, South Australia}
\end{figure}
\end{center}
\tiny{Fuente Australian landforms by Twidale \& Campbell (2005)}
\end{frame}
%%%%%%%%%%%%%%%%%%%%%%%%%%%%%%%%%%%%%%%%%%%%%%%%%%%%%%%%%%%%%%%%%%%%
\begin{frame}
\frametitle{Modelado Granítico - Estructural}
\framesubtitle{Rocas Penitent, monkstones, tombstones}
\begin{center}
\begin{figure}
\includegraphics[scale=0.55]{penitent}
\caption{Tungkillo, eastern Mt Lofty Ranges}
\end{figure}
\end{center}
\tiny{Fuente Australian landforms by Twidale \& Campbell (2005)}
\end{frame}
%%%%%%%%%%%%%%%%%%%%%%%%%%%%%%%%%%%%%%%%%%%%%%%%%%%%%%%%%%%%%%%%%%%%
\end{document}